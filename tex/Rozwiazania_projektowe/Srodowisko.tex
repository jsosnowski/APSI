\subsection{Środowisko}

Nowoczesne sklepy internetowe budowane są według kilku znanych deweloperom
zasad. Można stwierdzić, że zostały już wypracowane różnego rodzaju standardy i
wzorce, które pomagają z jednej strony deweloperom przyśpieszyć i ułatwić proces
implementacji a z drugiej pozwalają klientom w łatwiejszy sposób zorientować się
w mechanizmach działania sklepu, który zbudowany jest na tych samych zasadach
jak podobne odwiedzane wcześniej przez klienta. 

Środowiskiem pracy, zarówno dla klientów jak i dla pracowników firmy, będzie
przeglądarka internetowa. Rozwiązanie to podyktowane jest chęcią umożliwienia
dostępu do zasobów sklepu z dowolnego miejsca na Ziemi i za pomocą dowolnego
sprzętu posiadającego dostęp do Internetu. Rozwiązania takie jak dedykowane
aplikacje mogą być przydatne na niektórych rodzajach urządzeń, jednak tworzenie
ich na wszystkie możliwe rynki (stacjonarne, mobilne itp.) stanowiłoby duże
wyzwanie i spowodowało znaczące przekroczenie zarówno budżetu jak i
harmonogramu. 

Zdecydowano się na wsparcie następujących rodzajów przeglądarek:
\begin{enumerate}
  \item Google Chrome (od wersji 17 wzwyż)
  \item Mozilla Firefox (od wersji 11 wzwyż)
  \item Safari (od wersji 4 wzwyż)
  \item Internet Explorer (od wersji 7 wzwyż)
\end{enumerate}

Pozostałe przeglądarki także powinny poprawnie prezentować stronę internetową
sklepu, jednak wsparcie dla nich nie jest wymaganiem a co za tym idzie, dla
przeglądarek tych nie będą przeprowadzane testy. 

Wygląd strony internetowej powinien być taki sam (z różnicami maksymalnie 0.04%
zawartości) dla każdej przeglądarki internetowej. Ewentualne różnice wynikające
na przykład z różnicy w formatach monitorów czy ich wielkości powinny być
obsługiwane przez mechanizmy wewnętrzne.

Ewentualne aplikacja wspomagające korzystanie ze sklepu (na przykład
zdobywające coraz większą popularność aplikacje na urządzenia mobilne) nie
znajdują się w fazie analizy w niniejszym projekcie, ewentualnie mogą zostać
stworzone w czasie rozbudowy i utrzymywania systemu. Aby pozostawić możliwość
tego rodzaju rozszerzeń należy zadbać o odpowiedni protokół komunikacyjny
uniezależniający działanie serwerów aplikacyjnych i bazy danych od klienta,
który dostarcza dane i polecenia.

