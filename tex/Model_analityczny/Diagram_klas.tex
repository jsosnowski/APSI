\subsection{Diagram klas}


\subsection{Diagram Klas}

Przedstawiony na poniższym obrazku diagram klas reprezentuje wszystkie
wykorzystywane przez Zleceniodawcę elementy składające się na cały system.
Diagram ten ma znaczenie przede wszystkim dla deweloperów i osób zajmujących się
wytwarzaniem oprogramowania, tym niemniej powinien zostać zatwierdzony przed
przedstawicieli Zleceniodawcy - diagram klas jest bowiem punktem łączącym z
jednej strony wyobrażenie klienta o podziale funkcjonalności a z drugiej decyzje
projektowe podjęte przez zespół zajmujący się implementacją.

Diagram klas powinien obrazować zależności (agregacje, kompozycje, relację
dziedziczenia) pomiędzy poszczególnymi klasami na tyle szczegółowo, by osoby 
nieposiadające wykształcenia informatycznego i nieznające metod programowania
obiektowego mogłby zrozumieć zasadę podziału bez szczegółowych wyjaśnień. Z tego
też powodu na poniższym rysunku skoncentrowano się na powiązaniach pomiędzy
poszczególnymi klasami a nie na nazywaniu i przedstawianiu atrybutów i metod
poszczególnych klas. Nie stanowią one żadnej wartości z punktu widzenia
Zleceniodawcy a mogą stanowić ograniczenie i usztywnienie schematu dla
deweloperów, którzy lepiej znają metody dostarczania funkcjonalności i będą
mogli lepiej modyfikować schemat w zależności od potrzeb, nie naruszając
jednocześnie warunków umowy. Wszystkie atrybuty czy operacje ważne z punktu
widzenia Zleceniodawcy, które mogą mieć wpływ na ocenę projektu zostały
umieszczone na diagramie.

\includegraphics[width=\textwidth,
height=0.8\textheight]{graphics/ClassDiagram.png}


\newpage
Opis klas na przedstawionym diagramie:

\begin{description}
	\item[Użytkownik] \hfill \\
		Klasa abstrakcyjna, będąca bazową dla klas Klient i Pracownik, przechowuje
		informacje dotyczące danej osoby - imię, nazwisko, adres e-mail itp.
	\item[Pracownik] \hfill \\
	 	Osoba z obsługi sklepu, odpowiedzialna za realizację i zarządzanie
	 	zamówieniami
	\item[Stały klient] \hfill \\
		Osoba charakteryzująca się dużą liczbą zamówień bądź długim czasem obecności
		na stronie (czas liczony od czasu rejestracji)
	\item[Klient] \hfill \\
		Osoba składająca zamówienia w sklepie, edytująca swoje zamówienia i opłacająca
		je
	\item[Typ pracownika] \hfill \\
		Enumeracja, będąca oznaczeniem rodzaju pracownika (Szeregowy Pracownik,
		Kierownik itp.)
	\item[Urlop] \hfill \\
		Obsługa urlopów dla pracowników pod kątem czasu ich trwania, momentu ich
		rozpoczęcia (i zakończenia) itp.
	\item[Uprawnienia] \hfill \\
		Obsługa uprawnień zarówno dla pracowników jak i klientów. Pozwala na
		ustalanie, kto ma jakie uprawnienia do edycji i podglądu danych
	\item[Wiadomość] \hfill \\
		Treści przesyłane pomiędzy pracownikami i klientami, służące do przekazywania
		informacji na temat zamówień
	\item[Typ wiadomości] \hfill \\
		Enumeracja, jaki rodzaj wiadomości jest przekazywany (Zapytanie, Edycja
		Zamówienia itp.)
	\item[Opinia] \hfill \\
		Tekst na temat zamówienia, ocena poprawności i jakości realizacji zamówienia
	\item[Typ opinii] \hfill \\
		Enumeracja, jaka opinia została wydana (Pozytywna, Negatywna, Neutralna)
	\item[Zamówienie] \hfill \\
		Informacje na temat złożonego przez klienta Zamówienia
	\item[Uwagi do zamówienia] \hfill \\
		Wszelkiego rodzaju informacje, jakie klient chce zawrzeć w momencie złożenia
		zamówienia - na przykład zaznaczanie wysyłki jako prezent, ustalenie, przed
		jakim terminem zamówienie nie powinno być wysłane, czy możliwy jest odbiór
		osobisty itp.
	\item[Sposób płatności] \hfill \\
		Informacja, jak użytkownik chce zapłacić za złożone zamówienie - inaczej
		wygląda procesowanie zapłaty kartą (wysłanie następuje dopiero po wpłynięciu
		pieniędzy, opłata za pobraniem jest uiszczana dopiero po wysłaniu)
	\item[Część] \hfill \\
		Pojedyncza część rowerowa wraz z informacjami na jej temat - rozmiar, nazwa,
		cena itp.
	\item[Zestaw] \hfill \\
		Złożenie kilku części w jeden, funkcjonalnie sprawny rower. Przechowuje
		informację o tym, jakie części są wymagane, ile ma ich być (rama - 1, pedały
		-2, przerzutki - nieokreślone)
	\item[Kategoria] \hfill \\
		Informacja, do jakiej kategorii zaliczana jest dana część. Jest to pomocne do
		układania zestawów i sprawdzania ich poprawności
	\item[Dostawa] \hfill \\
		Informacje na temat jednej dostawy, jakie części i w jakiej ilości zostały
		dostarczone i kiedy
	\item[Firma dostawcza] \hfill \\
		Informacje na temat firmy, która dostarcza części - dane kontaktowe, adres
		oraz jakie części są w stanie dostarczyć
	\item[Magazyn] \hfill \\
		Klasa pozwalająca na zarządzanie częściami przechowywanymi w magazynie,
		sprawdzanie ich dostępności oraz aktualizacja stanu
	\item[Interfejs graficzny] \hfill \\
		Klasa będąca ``wejściem'' do diagramu klas, odpowiedzialna za podstawową,
		wstępną integrację z użytkownikiem
\end{description}