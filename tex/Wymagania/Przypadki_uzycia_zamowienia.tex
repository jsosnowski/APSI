\subsubsection{Opis przypadków użycia - zamowienia}

Przypadki użycia wyjaśniające funkcjonalności systemu związane z zarządzaniem
zamówieniami.

\begin{enumerate}
  \item Prezentacja zamówień\\
  \begin{tabularx}{\linewidth}{ c X }
  Aktor: & Pracownik \\
  Opis: & Prezentacja panelu z listą wszystkich zamówień znajdujących się w
  systemie oraz możliwościami kontroli i zarządzania nimi.\\
  \end{tabularx}
	\begin{enumerate}
	  \item Pracownik loguje się w Panelu Zarządzania
	  \item Wybiera Panel Zarządzania Zamówieniami
	  \item Wyświetlana jest lista zamówień z możliwością modyfikacji widoków
	  oraz panelem opcji (wszystkie opisane w poniższych przypadkach użycia)
	\end{enumerate}

  \item Edycja, modyfikacja zamówień\\
  \begin{tabularx}{\linewidth}{c X}
  Aktor: & Pracownik \\
  Opis: & Funkcjonalność umożliwia modyfikację produktów w zamówieniu oraz
  zmianę danych odbiorcy.
  \end{tabularx}
	\begin{enumerate}
	  \item Pracownik po autoryzacji w panelu sterowania systemu, przechodzi do
	  panelu Zarządzania Zamówieniami (patrz Zamowienia przypadek użycia 1)
	  \item Z wyświetlonej przez system listy zamówień, pracownik wybiera jeden
	  element
	  \item W celu edycji produktów:
		\begin{enumerate}
		  \item Wybiera opcję podglądu zawartości zamówienia
		  \item Z wyświetlonej listy zamówionych produktów zaznacza jedną pozycję
		  \item Wybiera opcję edycji
		  \item Otrzymuje informacje o konkretnym produkcie (jego ID, szczegółowy opis)
		  oraz zamówioną ilość oraz podsumowanie (cenę, informację o udzielonych rabatach na dany produkt)
		  \item Pole z ilością produktu umożliwia modyfikację – wystarczy wprowadzić
		  liczbę z zakresu od 1 do maksymalnej liczby aktualnie dostępnych produktów w
		  magazynie (0 nie wchodzi w zakres bo do tego służy funkcja usunięcia)
		\end{enumerate}
	  \item W celu usunięcia produktu:
		\begin{enumerate}
		  \item Wybiera opcję podglądu zawartości zamówienia
		  \item Z wyświetlonej listy zamówionych produktów zaznacza jedną pozycję
		  \item Wybiera opcję Usuń
		  \item System pyta o potwierdzenie i po akceptacji dokonuje wykluczenia
		  produktu z zamówienia oraz wysyła powiadomienie do Zamawiającego
		\end{enumerate}
	  \item W celu dodania produktu:
		\begin{enumerate}
		  \item Wybiera opcję podglądu zawartości zamówienia
		  \item Wybiera opcję Dodaj produkt
		  \item Otworzony zostaje system zakupowy\\ 
		  (przebieg wyboru produktu - opisany w przypadkach użycia odnoszących się do
		  Produktów)
		  \item Po wybraniu produktu system wyświetla informację o tym jakie zostaną
		  wprowadzone zmiany i czeka na akceptację
		  \item Po akceptacji, zamówienie zostaje zmodyfikowane (produkt dodany),
		  koszt zaktualizowany oraz system informuje odbiorcę zamówienia (klienta) o
		  zaszłych zmianach – za pomocą wiadomości email (z ewentualną poprawioną
		  fakturą pro-forma, jeśli była zaznaczona taka opcja) 
	  \end{enumerate} %koniec alternatywy Dodania produktu
	\end{enumerate} %koniec UC Edycja, modyfikacja zamówień
  
  \item Zmiana danych zamawiającego\\
  \begin{tabularx}{\linewidth}{c X}
  Aktor: & Pracownik \\
  Opis: & Można zmienić dane odbiorcy na potrzeby danego zamówienia (zmiana
  danych tylko w ramach konkretnej faktury). Dotyczy to w szczególności adresu i
  danych osobowych osoby odpowiedzialnej za zamówienie.
  \end{tabularx}  
	\begin{enumerate}
	  \item Pracownik po autoryzacji w panelu sterowania systemu, przechodzi do
	  panelu Zarządzania Zamówieniami (patrz Zamówienia przypadek użycia 1)
	  \item Z wyświetlonej przez system listy zamówień, pracownik wybiera jeden
	  element i wybiera opcję Zmień Dane Odbiorcy
	  \item System prezentuje aktualnie dane odbiorcy (mogą to być aktualne dane
	  klienta, albo już wcześniej modyfikowane dane osobowe wprowadzone specjalnie w
	  ramach tego zamówienia)
	  \item Pracownik modyfikuje wybraną przez siebie składową danych (wszystkie
	  elementy pozwalają na edycję) i zatwierdza wprowadzone zmiany
	  \item System wyświetla zapytanie o potwierdzenie zmian i po jego akceptacji
	  wysyła powiadomienie do klienta o zaszłych zmianach – wiadomość drogą
	  elektroniczną
	\end{enumerate}

  \item Usunięcie zamówienia w całości\\
  \begin{tabularx}{\linewidth}{c X}
  Aktor: & Pracownik \\
  Opis: & Istnieje możliwość anulowania zamówienia – na życzenie klienta lub z
  powodów biznesowych sklepu.
  \end{tabularx}  
	\begin{enumerate}
	  \item Pracownik po autoryzacji w panelu sterowania systemu, przechodzi do
	  panelu Zarządzania Zamówieniami (patrz Zamówienia przypadek użycia 1)
	  \item Z wyświetlonej przez system listy zamówień, pracownik wybiera jeden
	  element i wybiera opcję Usuń zamówienie
	  \item System wyświetla ostrzeżenie (wraz ze szczegółową informacją o
	  zamówieniu) i pyta o potwierdzenie
	  \item Pracownik potwierdza chęć usunięcia danego zamówienia. Ma też możliwość
	  wpisania krótkiego uzasadnienia tej operacji
	  \item System dokonuje usunięcia oraz wysyła powiadomienie o anulowaniu
	  zamówienia do zamawiającego (drogą elektroniczną)
	\end{enumerate}

  \item Edycja formy płatności\\
  \begin{tabularx}{\linewidth}{c X}
  Aktor: & Pracownik \\
  Opis: & Pracownik ma możliwość zmiany początkowo wybranej formy płatności
  danego zamówienia. Odbywa się to na wniosek zamawiającego lub osoby
  odpowiedzialnej za zamówienia po stronie Sklepu.
  \end{tabularx}    
	\begin{enumerate}
	  \item Z listy zamówień pracownik wybiera jedno i wybiera opcję Zmiana formy
	  płatności
	  \item System prezentuje widok wyboru pomiędzy dostępnymi formami płatności
	  (specyfikacja w wymaganiach niefunkcjonalnych)
	  \item Pracownik dokonuje wyboru formy oraz waluty.
	  \item System powiadamia klienta o zmianie formy płatności drogą elektroniczną.
	\end{enumerate}

  \item Wybór sposobu potwierdzenia zamówienia\\
  \begin{tabularx}{\linewidth}{c X}
  Aktor: & Pracownik \\
  Opis: & Możliwość zmiany sposobu udokumentowania przeprowadzonej transakcji
  (zazwyczaj będzie to faktura albo paragon). Powodem takich zmian mogą być
  nawet regulacje prawne.
  \end{tabularx}	
	\begin{enumerate}
	  \item Z listy zamówień pracownik wybiera jedno i wybiera opcję Zmiana
	  Potwierdzenia Transakcji
	  \item System prezentuje widok wyboru pomiędzy dostępnymi sposobami
	  potwierdzenia (udokumentowania) prowadzonej transakcji.
	  \item Pracownik dokonuje wyboru oraz może uruchomić proces generacji
	  dokumentu.
	  \item W przypadku generacji dokumenty system wyświetla go pracownikowi.
	  \item Po akceptacji informacje o zmianie wraz z dokumentami wysyłane są drogą
	  elektroniczną do klienta.
	\end{enumerate}

  \item Generowanie faktury pro-forma dla danego zmówienia\\
  \begin{tabularx}{\linewidth}{c X}
  Aktor: & Pracownik \\
  Opis: & Możliwość utworzenie faktury pro-forma na podstawie danego zamówienia
  oraz przesłanie jej klientowi drogą elektroniczną lub wydruk.
  \end{tabularx}
	\begin{enumerate}
	  \item Z listy zamówień pracownik wybiera jedno i wybiera opcję Generuj
	  Pro-Forma
	  \item Dla wybranego zamówienia system generuje pełną fakturę po czym
	  prezentuje ją pracownikowi
	  \item Pracownik ma możliwość odrzucenia lub akceptacji dokumentu.
	  \item W przypadku akceptacji system wyświetla widok wyboru z opcjami wysyłki
	  do klienta.
	  \item Po wybraniu pożądanej opcji przez pracownika, system wysyła dokument do
	  klienta albo do drukarki.
	\end{enumerate}

  \item Zarządzanie terminem dostawy\\
  \begin{tabularx}{\linewidth}{c X}
  Aktor: & Pracownik \\
  Opis: & Termin realizacji zamówienia może być modyfikowany dowolnie w
  zależności od możliwości biznesowych Sklepu.
  \end{tabularx}
	\begin{enumerate}
	  \item Z listy zamówień pracownik wybiera jedno i wybiera opcję Edycji Daty
	  Realizacji
	  \item System prezentuje widok kalendarza z zaznaczoną dotychczasową datą
	  realizacji.
	  \item Pracownik przesuwa datę realizacji projektu i ma możliwość podania
	  wiadomości wyjaśniającej modyfikację.
	  \item System wysyła powiadomienie o zmianie terminu do klienta wraz z
	  informacją wyjaśniającą wpisaną przez pracownika.
	\end{enumerate}

  \item Ustawianie aktualnego stanu zamówienia\\
  \begin{tabularx}{\linewidth}{c X}
  Aktor: & Pracownik \\
  Opis: & Zamówienie może znajdować się w pewnych stanach realizacji (np. w
  przygotowaniu, w realizacji, wysłane - konkretne stany określają wymagania
  niefunkcjonalne). Istnieje możliwość zmiany aktualnego stanu zamówienia.
  \end{tabularx}
	\begin{enumerate}
	  \item Z listy zamówień pracownik wybiera jedno i wybiera opcję Zmień Stan
	  \item System prezentuje widok z dostępnymi stanami dla danego zamówienia
	  \item Pracownik dokonuje wyboru i zatwierdza zmiany.
	  \item Jeśli pracownik wybiera opcję Powiadom, to system powiadamia klienta o
	  zmianie stanu jaka nastąpiła i przesyła krótkie wyjaśnienie.
	\end{enumerate}
	 
\end{enumerate}
