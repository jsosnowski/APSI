\newpage
\subsection{Wymagania niefunkcjonalne}

\begin{enumerate}
  \item Pojemność: \\ 
  System powinien mieć możliwość przechowywania danych o 100 tys.
  użytkowników 
  \item Wydajność: \\ 
  System powinien obsługiwać bez znaczącego spadku wydajności 400
  użytkowników ``jednocześnie''. Zakładając, że użytkownik będzie wymagał
  maksymalnie 20 odświeżeń widoku systemu na minutę (jedna podstrona na 3
  sekundy). System powinien działać z wydajnością 8000 odświeżeń/minutę. 
  \item System powinien być dostępny dla klientów 24 godziny na dobę 7 dni w
  tygodniu (możliwe są przerwy konserwacyjne, jednak nie dłuższe niż 4 godziny na miesiąc pracy) 
  \item Średni czas naprawy (MTTR - ang. Mean Time to Recover) na poziomie
  1~godziny
  \item System powinien umożliwiać klientom dostęp z dowolnego miejsca na
  świecie za pomocą sieci Internet oraz jego działanie powinno być niezależne od
  używanej platformy systemowo-sprzętowej użytkownika.
  \item Dane osobowe muszą być przetwarzane zgodnie z ustawą o ochronie danych
  osobowych z dnia 29 sierpnia 1997 r. 
  \item Klient powinien mieć dostęp do wszystkich swoich danych (łącznie z
  możliwością ich aktualizacji i usunięcia) zgodnie z polskim prawem
  \item Dane te powinny być chronione w zależności od ich poziomu poufności
  (dane do autoryzacji powinny być zabezpieczone przed możliwością odczytu nawet
  przez administratora) 
  \item Komunikacja pomiędzy klientem (przeglądarką internetową, aplikacją
  mobilną itp.) powinna być szyfrowana w sposób uniemożliwiający odczytanie czułych informacji
  \item System powinien mieć wbudowane procedury przeciwdziałania sytuacjom
  awaryjnym - procedury uruchamiane przez administratora
  \begin{enumerate}
    \item Procedury sprawdzenia spójności danych - po odzyskaniu sprawności, np.
    po awarii sprzętu
    \item Procedury uruchamiane w przypadku wykrycia włamania (między innymi,
    odłączenie systemu od sieci Internet, zablokowanie modyfikacji elementów
    systemu itp.)
  \end{enumerate} 
  \item System posiadać będzie hierarchię uprawnień (ról) dla użytkowników,
  przydzielanych im w celu umożliwienia korzystania z dodatkowych
  funkcjonalności
  \item System domyślnie powinien nadawać użytkownikowi uprawnienia nie większe
  niż niezbędne mu do poprawnego zamawiania produktów i zarządzania swoim
  kontem
  \item System powinien umożliwiać automatyczne wysyłanie klientowi wiadomości
  e-mail (z prośbą o potwierdzenie zmiany hasła czy akceptacji warunków rejestracji)
  \item System powinien umożliwiać użytkownikowi zmianę (w ograniczonym stopniu)
  już złożonego zamówienia (zmiana adresu przed wysyłką itp.) bez konieczności
  ingerencji pracownika sklepu
  \item System powinien być zdolny do wyświetlania informacji w wielu językach.
  Początkowo będzie to język polski i angielski. Istnieje jednak możliwość
  rozszerzenia o kolejne.
  \item \label{itm:OPD} Okres Przechowywania Danych - to czas przez który będą
  przechowywane dane użytkownika sprzed ich zmiany lub wyrejestrowania - system zapewnia
  magazynowanie tych danych co najmniej przez 7 dni.
  \item \label{itm:OMD} Okres Magazynowania Danych to czas 30 dni przez które
  system powinien przechowywać dane o klientach po usunięciu konta klienta (czas ten może się
  zmienić z powodów prawnych)
  \item \label{itm:Platnosci} System wspiera następujące formy płatności:
  \begin{enumerate}
    \item Płatność gotówką
    \item Przelew bankowy
    \item Płatność ratalna w oparciu o zewnętrzną usługę bankową
  \end{enumerate}
  Waluty: polski złoty, euro, bitcoin
  \item \label{itm:PotwierdzenieTransakcji} Forma potwierdzenia transakcji:
  faktura albo paragon
  \\
  System powinien umożliwiać generowanie tych dokumentów oraz ich wydruk.
  
\end{enumerate}