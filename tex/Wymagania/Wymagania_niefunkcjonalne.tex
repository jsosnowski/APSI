\newpage
\subsection{Wymagania niefunkcjonalne}

\begin{enumerate}
  \item System powinien mieć możliwość przechowywania danych o 100 tys.
  użytkowników 
  \item System powinien obsługiwać bez znaczącego spadku wydajności 400
  użytkowników jednocześnie 
  \item System powinien być dostępny dla klientów 24 godziny na dobę 7 dni w
  tygodniu (możliwe są przerwy konserwacyjne, jednak nie dłuższe niż 4 godziny na miesiąc pracy) 
  \item System powinien umożliwiać klientom dostęp z dowolnego miejsca na
  świecie za pomocą sieci Internet 
  \item Klient powinien mieć dostęp do wszystkich swoich danych 
  \item Dane te powinny być chronione w zależności od ich tajności (hasło -
  dostępne tylko w postaci wartości funkcji skrótu; adres, e-mail - dostępne konkretnemu klientowi i pracownikom) 
  \item Komunikacja pomiędzy klientem (przeglądarką internetową, aplikacją
  mobilną itp.) powinna być szyfrowana w sposób uniemożliwiający odczytanie czułych informacji
  \item Autoryzacja użytkowników powinna odbywać się za pomocą loginu i hasła
  znanych tylko konkretnemu użytkownikowi 
  \item System powinien nadawać użytkownikowi uprawnienia niezbędne mu do
  poprawnego zamawiania produktów i zarządzania swoim kontem, jednak nie większe 
  \item System powinien umożliwiać automatyczne wysyłanie klientowi wiadomości
  e-mail (z prośbą o potwierdzenie zmiany hasła czy akceptacji warunków rejestracji)
  \item System powinien umożliwiać użytkownikowi zmianę (w ograniczonym stopniu)
  już złożonego zamówienia (zmiana adresu przed wysyłką itp.) bez konieczności ingerencji pracownika sklepu 
  \item System powinien przechowywać dane o klientach przez co najmniej 30 dni
  po wyrejestrowaniu lub usunięciu klienta (czas ten może się zmienić z powodów prawnych)
  
\end{enumerate}