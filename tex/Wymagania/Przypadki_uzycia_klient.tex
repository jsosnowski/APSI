\subsubsection{Opis przypadków użycia - klient}

Opis przypadków użycia wyjaśniające funkcjonalności związane z zarządzaniem
klientami:

\begin{enumerate}
  \item Rejestracja klienta \\
  \begin{tabularx}{\linewidth}{ c X }
  Aktor: & Klient \\
  Opis: & Możliwość rejestracji nowego klienta.\\
  \end{tabularx}
   \begin{enumerate}
    \item Klient uruchamia stronę internetową sklepu i wybiera opcję rejestracji
    \item Klient wstawia swoje dane osobowe i wybiera domyślny model płatności
    (kartą, za pobraniem itp.)
    \item System sprawdza wstawione dane (takie same hasła, czy istnieje już
    zarejestrowany w systemie użytkownik, czy istnieje podany adres e-mail itp.)
    \item System wysyła e-mail powitalny na adres podany przez klienta
    \item W ciągu określonego, zdefiniowanego czasu klient wybiera przesłany w
    e-mailu link, stają się pełnoprawnym użytkownikiem sklepu
  \end{enumerate}
  \item Złożenie zamówienia \\
  \begin{tabularx}{\linewidth}{ c X }
  Aktor: & Klient \\
  Opis: & Przedstawienie sposobu złożenia zamówienia.\\
  \end{tabularx}
  \begin{enumerate}
    \item Klient uruchamia stronę internetową sklepu i wyszukuje interesujące go
    produkty
    \item W momencie znalezienia pasującego produktu użytkownik wybiera opcję
    dodania do koszyka
    \item Po zakończeniu wyszukiwania użytkownik wybiera opcję przejścia do kasy
    \item System sprawdza, czy użytkownik jest zalogowany. Jeśli nie, procesuje
    przypadek użycia Logowanie do Systemu
    \item System sprawdza, czy użytkownik jest stałym klientem. Jeśli tak,
    dolicza rabat do ustalonej ceny (do sumy cen poszczególnych produktów)
    \item Użytkownik wybiera sposób płatności
    \item System dodaje do wcześniej ustalonej ceny koszty wynikające ze sposobu
    płatności
    \item Użytkownik wybiera sposób dostawy (poczta, kurier, odbiór osobisty
    itp.)
    \item System dodaje do ceny koszty wynikające ze sposobu dostawy
    \item Użytkownik, po sprawdzeniu wszystkich danych, decyduje się na złożenie
    zamówienia - po tym momencie nie może już ono być cofnięte
    \item System wysyła do użytkownika e-mail potwierdzający wraz z przewidywaną
    datą realizacji zamówienia
  \end{enumerate} 
  \item Edycja danych klienta \\
  \begin{tabularx}{\linewidth}{ c X }
  Aktor: & Klient \\
  Opis: & Możliwość zmiany, uzupełnienia danych osobowych klienta.\\
  \end{tabularx}
  \begin{enumerate}
    \item Klient uruchamia witrynę internetową sklepu
    \item Klient loguje się do systemu (tylko osoba zalogowana może zmieniać
    swoje dane)
    \item Klient edytuje wybrane pozycje ze swojego opisu (adres, numer
    telefonu itp.)
    \item W przypadku zmiany hasła klient proszony jest o podanie starego jak i
    nowego (dwukrotnie) hasła
    \item Klient zatwierdza wprowadzone zmiany
    \item System wysyła na podany przez użytkownika adres e-mail (nowy, jeśli
    to adres e-mail był jedną ze zmienianych wartości) informację o zmianie.
  \end{enumerate}
  \item Wyrejestrowanie się klienta \\
  \begin{tabularx}{\linewidth}{ c X }
  Aktor: & Klient \\
  Opis: & Klient ma możliwość w każdym momencie usunąć swoje konto z systemu.\\
  \end{tabularx}
  \begin{enumerate}
    \item Klient uruchamia witrynę internetową i loguje się na swoje konto
    (przypadek użycia Logowanie Do Systemu)
    \item Klient wybiera opcję usunięcia danych
    \item System sprawdza, czy istnieją niezrealizowane (oczekujące) zamówienia.
    Jeśli tak, wyświetla się alert z informacją, czy dane zamówienie zostało już
    wcześniej opłacone
    \item Jeśli istniały już zamówienia, które zostały opłacone a nie zostały
    jeszcze zrealizowane, system zleca odesłanie określonej kwoty pieniężnej z
    powrotem na konto użytkownika (z pominięciem kosztów obsługi)
    \item Klient zostaje poproszony o podanie przyczyn swojej decyzji -
    wypełnianie jest nieobowiązkowe
    \item Dane przechowywane są przez następne 7 dni (wymaganie prawne). W tym
    czasie klient może ponownie zarejestrować się w systemie bez utraty
    poprzednich danych
    \item W przypadku braku ponownej rejestracji dane zostają na stałe usunięte
    z firmowej bazy danych
  \end{enumerate}
  \item Usunięcie klienta \\
  \begin{tabularx}{\linewidth}{ c X }
  Aktor: & Pracownik \\
  Opis: & Klienta można usunąć administracyjnie na przykład z powodów
  naruszenia regulaminu.\\
  \end{tabularx}
  \begin{enumerate}
    \item Pracownik sklepu wyszukuje klienta o konkretnym imieniu i nazwisku
    (lub według innych kryteriów)
    \item Pracownik wybiera opcję usunięcia klienta. 
    \item Pracownik wpisuje powód, dla którego usuwa użytkownika (informacja ta
    będzie przesłana do klienta w wiadomości e-mail)
    \item Pracownik wypełnia dane dotyczące kwestii niezrealizowanych zamówień i
    nieotrzymanych płatności
    \item Obie informacji (z poprzednich 2 kroków) są przekazywane na podany
    przez użytkownika adres e-mail
    \item Dane są przechowywane przez następne 30 dni - w tym czasie użytkownik
    może złożyć reklamację i ewentualnie odzyskać dostęp do konta
    \item Po 30 dniach, jeśli nie zostanie pozytywnie rozpatrzona prośba o
    przywrócenie konta, dane są na stałe usuwane z bazy danych
  \end{enumerate}
\end{enumerate}