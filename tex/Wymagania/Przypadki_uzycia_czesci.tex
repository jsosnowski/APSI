\subsubsection{Opis przypadków użycia - części}

Opis przypadków użycia wyjaśniające funkcjonalności związane z zarządzaniem
częściami:

\begin{enumerate}
  \item Lista części \label{lista-czesci} \\
  \begin{tabularx}{\linewidth}{ c X }
  Aktor: & Klient \\
  Opis: & Możliwość wyświetlenia listy dostępnych części.\\
  \end{tabularx}
   \begin{enumerate}
    \item Klient uruchamia stronę internetową sklepu i wybiera opcję przeglądania listy części
    \item System prezentuje listę części możliwych do kupienia.
  \end{enumerate}
  
  \item Wyszukiwanie części \label{wyszukiwanie-czesci} \\
  \begin{tabularx}{\linewidth}{ c X }
  Aktor: & Klient \\
  Opis: & Możliwość wyszukania konkretnych części z listy wszystkich dostępnych części.\\
  \end{tabularx}
   \begin{enumerate}
    \item Tak jak w punkcie \ref{lista-czesci}
    \item Klient wpisuje w wyszukiwarkę kryteria (z zakresu: kod, nazwa, opis, cena jednostkowa), według których części mają być wyszukane.
    \item System wyszukuje części według zadanych kryteriów i prezentuje listę.
  \end{enumerate}
  
  \item Dodanie części \label{dodanie-czesci} \\
  \begin{tabularx}{\linewidth}{ c X }
  Aktor: & Pracownik \\
  Opis: & Możliwość dodania nowego typu części.\\
  \end{tabularx}
   \begin{enumerate}
    \item Pracownik loguje się do Panelu Zarządzania Częściami i wybiera opcję dodania nowej części.
    \item System prezentuje pracownikowi formatkę dodania nowej części, z możliwością wypełnienia następujących atrybutów części:
    \begin{enumerate}
      \item Kod części (generowany automatycznie, z możliością edycji przez pracownika)
      \item Nazwa części
      \item Opis części
      \item Zdjęcie części (w jedym z popularnych formatów graficznych, takich jak JPG, PNG czy GIF)
      \item Cena jednostkowa części
      \item Widoczność części dla klientów (czy klienci będą mogli zobaczyć część na liście części możliwych do kupienia)
      \item Minimalna liczba sztuk w magazynie
    \end{enumerate}
    \item Pracownik uzupełnia wymagane pola i zatwierdza operację.
    \item System dodaje część do bazy danych i informuje użytkownika o zakończeniu operacji.
  \end{enumerate}
  
  \item Usunięcie części \\
  \begin{tabularx}{\linewidth}{ c X }
  Aktor: & Pracownik \\
  Opis: & Możliwość usunięcia danego typu części z bazy danych.\\
  \end{tabularx}
   \begin{enumerate}
    \item Pracownik loguje się do Panelu Zarządzania Częściami i wybiera opcję wyświetlenia listy dostępnych części.
    \item System prezentuje listę części tak jak w punkcie \ref{lista-czesci} z możliwością wyszukiwania tak jak w punkcie \ref{wyszukiwanie-czesci}.
    \item Pracownik wyszukuje żądaną część i wybiera opcję jej usunięcia.
    \item System sprawdza, czy w magazynie znajdują się części danego typu - jeśli tak (na magazynie znajduje się co najmniej 1 sztuka danego typu części), informuje o tym pracownika w formie ostrzeżenia. W przypadku akceptacji przez pracownika tego ostrzeżenia następuje kontynuacja do następnego punktu, w przeciwnym przypadku pracownik wraca do Panelu Zarządzania Częściami.
    \item System prosi pracownika o potwierdzenie zamiaru usunięcia danego typu części.
    \item W przypadku potwierdzenia zamiaru przez pracownika, system usuwa dany typ części z magazynu.
    \item Powrót do Panelu Zarządzania Częściami.
  \end{enumerate}
  
  \item Modyfikacja części \\
  \begin{tabularx}{\linewidth}{ c X }
  Aktor: & Pracownik \\
  Opis: & Możliwość modyfikacji atrybutów danego typu części.\\
  \end{tabularx}
   \begin{enumerate}
    \item Pracownik loguje się do Panelu Zarządzania Częściami i wybiera opcję wyświetlenia listy dostępnych części.
    \item System prezentuje listę części z możliwością wyszukiwania tak jak w punkcie \ref{wyszukiwanie-czesci}
    \item Pracownik wyszukuje żądaną część i wybiera opcję jej modyfikacji.
    \item System prezentuje formatkę taką jak w punkcie \ref{dodanie-czesci} ale wypełnioną danymi części, której modyfikację wybrał pracownik.
    \item Pracownik modyfikuje dane i zatwierdza operację.
    \item System prosi pracownika o potwierdzenie zamiaru modyfikacji danego typu części.
    \item W przypadku potwierdzenia zamiaru przez pracownika, system modyfikuje atrybuty danego typu części.
    \item Powrót do Panelu Zarządzania Częściami.
  \end{enumerate}
  
  \item Modyfikacja ilości sztuk części \\
  \begin{tabularx}{\linewidth}{ c X }
  Aktor: & Pracownik \\
  Opis: & Możliwość modyfikacji ilości sztuk danej części znajdujących się aktualnie w magazynie.\\
  \end{tabularx}
   \begin{enumerate}
    \item Pracownik loguje się do Panelu Zarządzania Częściami i wybiera opcję wyświetlenia listy dostępnych części.
    \item System prezentuje listę części z możliwością wyszukiwania tak jak w punkcie \ref{wyszukiwanie-czesci}
    \item Pracownik wyszukuje żądaną część i wybiera opcję modyfikacji jej stanu.
    \item System prezentuje aktualną liczbę sztuk części i prosi o podanie nowej wartości.
    \item Pracownik wpisuje nową wartość i zatwierdza operację.
    \item System prosi pracownika o potwierdzenie zamiaru modyfikacji ilości sztuk danego typu części.
    \item W przypadku potwierdzenia zamiaru przez pracownika, system modyfikuje ilość sztuk danego typu części.
    \item Powrót do Panelu Zarządzania Częściami.
  \end{enumerate}
  
  \item Modyfikacja widoczności części dla klientów \\
  \begin{tabularx}{\linewidth}{ c X }
  Aktor: & Pracownik \\
  Opis: & Możliwość modyfikacji widoczności danej części dla klientów, czyli czy klienci będą mogli zobaczyć część na liście części możliwych do kupienia.\\
  \end{tabularx}
   \begin{enumerate}
    \item Pracownik loguje się do Panelu Zarządzania Częściami i wybiera opcję wyświetlenia listy dostępnych części.
    \item System prezentuje listę części z możliwością wyszukiwania tak jak w punkcie \ref{wyszukiwanie-czesci}
    \item Pracownik wyszukuje żądaną część i wybiera opcję modyfikacji jej widoczności.
    \item System prezentuje aktualną widoczność części i prosi o podanie nowej wartości (widoczna lub niewidoczna).
    \item Pracownik wpisuje nową wartość i zatwierdza operację.
    \item System prosi pracownika o potwierdzenie zamiaru modyfikacji widoczności danego typu części.
    \item W przypadku potwierdzenia zamiaru przez pracownika, system modyfikuje widoczność danego typu części.
    \item Powrót do Panelu Zarządzania Częściami.
  \end{enumerate}
  
  \item Modyfikacja minimalnej liczby sztuk części \\
  \begin{tabularx}{\linewidth}{ c X }
  Aktor: & Pracownik \\
  Opis: & Możliwość modyfikacji minimalnej liczby sztuk części, znajdujących się aktualnie w magazynie.\\
  \end{tabularx}
   \begin{enumerate}
    \item Pracownik loguje się do Panelu Zarządzania Częściami i wybiera opcję wyświetlenia listy dostępnych części.
    \item System prezentuje listę części z możliwością wyszukiwania tak jak w punkcie \ref{wyszukiwanie-czesci}
    \item Pracownik wyszukuje żądaną część i wybiera opcję modyfikacji jej minimalnej liczby sztuk.
    \item System prezentuje aktualną minimalną liczbę sztuk i prosi o podanie nowej wartości (liczba naturalna).
    \item Pracownik wpisuje nową wartość i zatwierdza operację.
    \item System prosi pracownika o potwierdzenie zamiaru modyfikacji minimalnej liczby sztuk części.
    \item W przypadku potwierdzenia zamiaru przez pracownika, system modyfikuje minimalną liczbę sztuk części.
    \item Powrót do Panelu Zarządzania Częściami.
  \end{enumerate}
  
  \item Generowanie zamówienia na dostawę części \label{generowanie-zamowienia} \\
  \begin{tabularx}{\linewidth}{ c X }
  Aktor: & Pracownik \\
  Opis: & Możliwość wygenerowania zamówienia na dostawę nowych części do magazynu.\\
  \end{tabularx}
   \begin{enumerate}
    \item Pracownik loguje się do Panelu Zarządzania Częściami i wybiera opcję generowania zamówienia na dostawę.
    \item System prezentuje widok generowania zamówienia na dostawę, umożliwiający:
    \begin{enumerate}
      \item Dodanie typu części do zamówienia.
      \item Usunięcie wcześniej wprowadzonego typu części z zamówienia.
      \item Prezentację listy już wprowadzonych części.
    \end{enumerate}
    \item System początkowo wypełnia listę tymi częściami, których ilość w magazynie spadła poniżej zadanego minimalnego poziomu. Części dodawane są w minimalnej ilości wystarczającej do tego, aby poziom ten został osiądnięty.
    \item Pracownik dodaje nowe części według następującego schematu:
    \begin{enumerate}
      \item Pracownik wybiera przycisk umożliwiający dodanie typu części do zamówienia.
      \item System prezentuje listę części z możliwością wyszukiwania tak jak w punkcie \ref{wyszukiwanie-czesci}
      \item Pracownik wybiera żądany typ części i wpisuje ilość sztuk (liczba naturalna większa od zera), jaka ma zostać dodana do zamówienia.
      \item Pracownik zatwierdza operację.
      \item Powrót do widoku generowania zamówienia.
    \end{enumerate}
    \item Po wprowadzeniu wszystkich informacji o zamówieniu, pracownik zatwierdza operację.
    \item System prosi pracownika o potwierdzenie zamiaru wygenerowania zamówienia.
    \item W przypadku potwierdzenia zamiaru przez pracownika, system generuje plik PDF z zamówieniem na dostawę.
    \item Pracownik pobiera plik i wysyła go do dostawcy.
    \item Powrót do Panelu Zarządzania Częściami.
  \end{enumerate}
  
  \item Wprowadzenie dostawy części do magazynu \\
  \begin{tabularx}{\linewidth}{ c X }
  Aktor: & Pracownik \\
  Opis: & Możliwość wprowadzenia do systemu informacji o dostawie nowych części do magazynu.\\
  \end{tabularx}
   \begin{enumerate}
    \item Pracownik loguje się do Panelu Zarządzania Częściami i wybiera opcję wprowadzenia dostawy.
    \item System prezentuje widok wprowadzania dostawy, umożliwiający:
    \begin{enumerate}
      \item Dodanie typu części do dostawy.
      \item Usunięcie wcześniej wprowadzonego typu części z dostawy.
      \item Prezentację listy już wprowadzonych części.
    \end{enumerate}
    \item Pracownik dodaje nowe części według następującego schematu:
    \begin{enumerate}
      \item Pracownik wybiera przycisk umożliwiający dodanie typu części do dostawy.
      \item System prezentuje listę części z możliwością wyszukiwania tak jak w punkcie \ref{wyszukiwanie-czesci}
      \item Pracownik wybiera żądany typ części i wpisuje ilość sztuk (liczba naturalna większa od zera), jaka ma zostać dodana do aktualnego stanu magazynu.
      \item Pracownik zatwierdza operację.
      \item Powrót do widoku wprowadzania dostawy.
    \end{enumerate}
    \item System umożliwia zaimportowanie danych o dostawie z wcześniej wygenerowanych zamówień na dostawę (patrz punkt \ref{generowanie-zamowienia})
    \item Po wprowadzeniu wszystkich informacji o dostawie, pracownik zatwierdza operację.
    \item System prosi pracownika o potwierdzenie zamiaru wprowadzenia dostawy.
    \item W przypadku potwierdzenia zamiaru przez pracownika, system modyfikuje stan magazynu zgodnie z danymi wprowadzonymi przez pracownika.
    \item Powrót do Panelu Zarządzania Częściami.
  \end{enumerate}
  
  \item Prezentacja stanu magazynu \label{prezentacja-stanu-magazynu} \\
  \begin{tabularx}{\linewidth}{ c X }
  Aktor: & Pracownik \\
  Opis: & Możliwość prezentacji stanu magazynu.\\
  \end{tabularx}
   \begin{enumerate}
    \item Pracownik loguje się do Panelu Zarządzania Częściami i wybiera opcję prezentacji stanu magazynu.
    \item System prezentuje listę części aktualnie znajdujących się w magazynie.
  \end{enumerate}
  
  \item Prezentacja zmian stanu magazynu w zadanym okresie czasowym \\
  \begin{tabularx}{\linewidth}{ c X }
  Aktor: & Pracownik \\
  Opis: & Możliwość wyświetlenia zmian stanu magazynu w zadanym okresie czasowym.\\
  \end{tabularx}
   \begin{enumerate}
    \item Tak jak w punkcie \ref{prezentacja-stanu-magazynu}
    \item Pracownik podaje okres czasowy, z jakiego zmiany stanu magazynu mają być zaprezentowane.
    \item System prezentuje listę zmian z zadanego okresu, czyli listę wszystkich operacji zmieniających liczbę sztuk części w magazynie, wraz z informacjami o tych operacjach (jaka część, ilość sztuk, data operacji).
    \item System prezentuje wykres zależności całościowego stanu magazynu (liczba wszystkich części) od daty (daty pochodzą z zadanego wcześniej okresu czasowego).
    \item Powrót do Panelu Zarządzania Częściami.
  \end{enumerate}
  
  \item Prezentacja zmian ilości sztuk danej części w zadanym okresie czasowym \\
  \begin{tabularx}{\linewidth}{ c X }
  Aktor: & Pracownik \\
  Opis: & Możliwość wyświetlenia zmian stanu magazynu w zadanym okresie czasowym.\\
  \end{tabularx}
   \begin{enumerate}
    \item Tak jak w punkcie \ref{prezentacja-stanu-magazynu}
    \item Pracownik wyszukuje żądaną część i wybiera opcję wyświetlenia zmian ilości jej sztuk.
    \item Pracownik podaje okres czasowy, z jakiego zmiany ilości sztuk danej części mają być zaprezentowane.
    \item System prezentuje listę zmian z zadanego okresu, czyli listę wszystkich operacji zmieniających liczbę sztuk danej części, wraz z informacjami o tych operacjach (ilość sztuk, data operacji).
    \item System prezentuje wykres zależności ilości sztuk danej części od daty (daty pochodzą z zadanego wcześniej okresu czasowego).
    \item Powrót do Panelu Zarządzania Częściami.
  \end{enumerate}
\end{enumerate}