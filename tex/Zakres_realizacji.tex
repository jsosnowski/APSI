\section{Zakres realizacji}

Niniejszy rozdział prezentuje specyfikację wymagań odnośnie sposobu realizacji
projektu. Nakreślając role poszczególnych zespołów oraz główne zadania jakie
mają realizować w ramach swoich prac.

\subsection{Realizacja projektu}

Projekt jest realizowany w siedzibie firmy Bike Shop sp. z.o.o.
Umożliwi to szybkie podejmowanie decyzji w przypadku powstawania
ewentualnych niejasności a także łatwiejsze reagowanie na przeszkody,
jakie pojawiają się w czasie procesu projektowania. 
Zleceniodawca zobowiązał się do delegacji doświadczonego pracownika, który zna
specyfikę działania firmy oraz jej cele biznesowe. Ta osoba będzie uczestniczyć
w projekcie bezpośrednio na etapie analizy.

Zgodnie z dokumentem załączonym do umowy Zleceniodawca zobowiązał się do
udostępnienia firmie implementującej system pomieszczenia, w których
możliwa będzie praca zespołu projektowego. Pomieszczenia takie
powinny charakteryzować się dostępem do szybkiego łącza
internetowego. 

\subsection{Wstępny zarys technologiczny}
Tworzony przez zleceniobiorcę system zostanie stworzony w logice
trójwarstwowej umożliwiającej łatwe i wydajne zarządzanie całością
przedsięwzięcia oraz umożliwiającej dalszą modyfikację i rozbudowę.
Podział na warstwy jest następujący:

\begin{itemize}
  \item Warstwa prezentacji odpowiada za część graficzną, reprezentację danych
  przechowywanych w systemie oraz za umożliwienie użytkownikowi przeglądania
  dostępnych produktów, a także złożenie zamówienia. Osobny moduł odpowiedzialny
  jest za dostęp do administracyjnych części systemu, dostępny wyłącznie dla
  pracowników firmy Bike Shop z.o.o.
  \item Serwer aplikacji zawierający logikę tworzonego systemu, odpowiedzialny
  za zarządzanie zamówieniami, komunikację pomiędzy bazą danych oraz interfejsem
  klienckim a także wykorzystanie infrastuktury internetowej w celu zwiększenia
  wydajności
  \item Baza danych przechowująca informacje na temat wszystkich produktów
  dostępnych w sklepie, klientów posiadających swoje konta oraz składanych przez
  nich zamówieniach.
\end{itemize}

Proces projektowy systemu zostanie oparty o dwa niezależne zespoły (analityczne
i projektowe)

\subsection{Analitycy - wymagania}
Zespół ten w ramach projektu zajmuje się prowadzeniem analizy biznesowej,
badaniem potrzeb klientów, projektuje rozwiązania dla systemu. Jest
odpowiedzialny za określanie wymagań (zarówno funkcjonalnych jak i
niefunkcjonalnych). Jego zadaniem jest też dbałość o ich prawidłową realizację.

Główne zadania:

\begin{itemize}
  \item Określenie wymagań stawianych przed systemem
  \item Tworzenie specyfikacji wymagań
  \item Tworzenie planu testów
  \item Analiza środowiska systemowego
  \item Tworzenie dokumentów projektowych
  \item Odpowiedzi na wątpliwości powstałe na etapie projektowania, czy
  implementacji
\end{itemize}

\subsection{Projektanci - wymagania}
Zespół jest odpowiedzialny za stworzenie architektury nowopowstającego
systemu oraz zapisanie jej w postaci dokumentacji technicznej. Wyniki prac tego
zespołu są niezbędne dla późniejszych etapów. W czasie implementacji służą jako
wsparcie dla programistów tworzących system.

Główne zadania:

\begin{itemize}
  \item Tworzenie projektu systemu informatycznego (oddzielnie projekt
  architektury i bazy danych)
  \item Wybór technologii i metod realizacji systemu
  \item Tworzenie dokumentacji technicznej wykorzystywanej podczas implementacji
  \item Bieżące dostosowywanie wymagań do postępów prac
\end{itemize}