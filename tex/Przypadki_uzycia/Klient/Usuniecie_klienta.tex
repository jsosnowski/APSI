\item Usunięcie klienta \\
  \begin{tabularx}{\linewidth}{ c X }
  Aktor: & Pracownik \\
  Opis: & Klienta można usunąć administracyjnie na przykład z powodów
  naruszenia regulaminu.\\
  \end{tabularx}
  \begin{enumerate}
    \item Pracownik sklepu wyszukuje klienta o konkretnym imieniu i nazwisku
    (lub według innych kryteriów)
    \item Pracownik wybiera opcję usunięcia klienta. 
    \item Pracownik wpisuje powód, dla którego usuwa użytkownika (informacja ta
    będzie przesłana do klienta w wiadomości e-mail)
    \item Pracownik wypełnia dane dotyczące kwestii niezrealizowanych zamówień i
    nieotrzymanych płatności
    \item Obie informacji (z poprzednich 2 kroków) są przekazywane na podany
    przez użytkownika adres e-mail
    \item Dane są przechowywane przez Okres Magazynowania Danych (patrz
    Wymagania Niefunkcjonalne punkt \ref{itm:OMD}) - w tym czasie użytkownik
    może złożyć reklamację i ewentualnie odzyskać dostęp do konta
    \item Po tym czasie, jeśli prośba o przywrócenie konta nie zostanie
    pozytywnie rozpatrzona, dane są na stałe usuwane z systemu
  \end{enumerate}