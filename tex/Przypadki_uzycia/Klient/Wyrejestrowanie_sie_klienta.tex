  \item Wyrejestrowanie się klienta \\
  \begin{tabularx}{\linewidth}{ c X }
  Aktor: & Klient \\
  Opis: & Klient ma możliwość w każdym momencie usunąć swoje konto z systemu.\\
  \end{tabularx}
  \begin{enumerate}
    \item Klient uruchamia witrynę internetową i loguje się na swoje konto
    (przypadek użycia Logowanie Do Systemu)
    \item Klient wybiera opcję usunięcia danych
    \item System sprawdza, czy istnieją niezrealizowane (oczekujące) zamówienia.
    Jeśli tak, wyświetla się alert z informacją, czy dane zamówienie zostało już
    wcześniej opłacone
    \item Jeśli istniały już zamówienia, które zostały opłacone a nie zostały
    jeszcze zrealizowane, system zleca odesłanie określonej kwoty pieniężnej z
    powrotem na konto użytkownika (z pominięciem kosztów obsługi)
    \item Klient zostaje poproszony o podanie przyczyn swojej decyzji -
    wypełnianie jest nieobowiązkowe
    \item Dane przechowywane są przez Okres Przechowywania Danych (wymaganie
    prawne - patrz Wymagania niefunkcjonalne punkt \ref{itm:OPD}). W tym
    czasie klient może ponownie zarejestrować się w systemie bez utraty poprzednich danych
    \item W przypadku braku ponownej rejestracji dane zostają na stałe usunięte
    z firmowej bazy danych
  \end{enumerate}