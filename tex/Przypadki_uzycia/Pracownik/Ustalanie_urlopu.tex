\item Ustalanie urlopu \\

	Opis słowny - w nowoczesnych firmach przyjaznych pracownikowi kwestia urlopu
	jest niezwykle ważna. Po pierwsze konieczne jest zapewnienie firmie rąk do
	pracy. Po drugie zadowolony pracownik jest najcennieszy, dlatego istotny jest
	kompromis pomiędzy dniami urlopowymi, a pracującymi (globalnie, dla
	wszystkich). Ta funkcja systemu umożliwia zgłaszanie chęci wykorzystania czasu
	urlopu - wyrażenie swoich preferencji urlopowych przez pracownika

  \begin{longtable}{|p{5cm}|p{7cm}|}
 	\hline
	\textbf{Aktor} & Pracownik \\
	\hline
	\textbf{Warunki początkowe} & Pracownik zalogowany, posiada niewykorzystane
	dni urlopowe\\
	\hline
	\textbf{Opis przebiegu interakcji} & Wybór panelu urlopów,
	zaznaczenie preferowanych terminów, oczekiwanie na akceptację \\
	\hline
	\textbf{Sytuacje wyjątkowe} & Niezgodność liczby dni urlopu w stosunku do
	zaznaczonego okresu preferowanego urlopu - wymagana korekta
	\\
	\hline
	\textbf{Warunki końcowe} & Wysłanie preferencji urlopowych do akceptacji przez
	kierownika
	\\
	\hline
 \end{longtable}
  

  \begin{tabularx}{\linewidth}{ c X }
  Aktor: & Pracownik \\
  \end{tabularx}
  \begin{enumerate}
    \item Pracownik uruchamia aplikację internetową sklepu i loguje się do systemu.
    \item Pracownik wybiera panel urlopów.
    \item System informuje pracownika o ilości dni urlopowych pozostałych do wykorzystania.
    \item Pracownik dodaje do kalendarza firmowego nowe żądanie urlopu.
    \item Pracownik uzupełnia dane dotyczące czasu przebywania na urlopie (datę początkową oraz datę końcową).
    \item System sprawdza, czy żądanie pracownika jest poprawne (np. czy pracownik może wziąć tak długi urlop).
    Jeśli nie, pracownik jest informowany o przyczynie błędu i musi ponownie wypełnić dane o urlopie.
    \item System zapisuje żądanie urlopu i informuje użytkownika o zmianie statusu żądania
    na~,,Oczekiwanie na odpowiedź kierownika''.
    \item Kierownik przegląda żądanie zgodnie ze scenariuszem ,,Przeglądanie żądań urlopowych''.
    \item Pracownik jest informowany o rozpatrzeniu żądania.
  \end{enumerate} 
  
