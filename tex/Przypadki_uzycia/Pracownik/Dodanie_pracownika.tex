  \item Dodawanie pracownika \\
  
  Opis słowny - ten przypadek użycia wspiera proces rozwoju firmy poprzez
  zatrudnianie nowych pracowników. Dane na temat wszystkich osób związanych ze
  sklepem są lepiej wykorzystywane jeśli są zarządzane przez system
  informatyczny. Aby uruchomić tą procedurę należy mieć specjalne uprawnienia
  jakie posiadają wyznaczeni pracownicy, czyli Kierownicy.
  
  \begin{longtable}{|p{5cm}|p{7cm}|}
 	\hline
	\textbf{Aktor} & Kierownik \\
	\hline
	\textbf{Warunki początkowe} & Kierownik zalogowany, posiadający wszelkie dane
	nowego pracownika\\
	\hline
	\textbf{Opis przebiegu interakcji} & Wybór panelu zarządzania sklepem,
	wypełnienie danych pracownika i potwierdzenie zapisu \\
	\hline
	\textbf{Sytuacje wyjątkowe} & Błędne dane, dany pracownik już zarejestrowany \\
	\hline
	\textbf{Warunki końcowe} & Zarejestrowanie nowego pracownika w systemie \\
	\hline
 \end{longtable}
  
  \begin{tabularx}{\linewidth}{ c X }
  Aktor: & Kierownik \\
  \end{tabularx}
   \begin{enumerate}
    \item Kierownik uruchamia stronę internetową panelu zarządzania sklepem
    i~wybiera opcję rejestracji.
    \item Kierownik wprowadza dane osobowe zatrudnianego pracownika.
    \item System sprawdza wstawione dane (czy istnieje już zarejestrowany
    w~systemie użytkownik, czy istnieje podany adres e-mail itp.)
    \item System wysyła na adres e-mail pracownika podany przez~kierownika
    wiadomość powitalną wraz z~linkiem umożliwiającym aktywowanie konta oraz
    ustalenie hasła.
    \item W ciągu określonego, zdefiniowanego czasu pracownik odwiedza stronę
    o~adresie przesłanym w~wiadomości powitalnej i ustala hasło dla konta.
  \end{enumerate}

	Dodawanie pracownika - scenariusz alternatywny - błąd danych
	\begin{enumerate}
	  \item Kroki 1-3 scenariusza głównego
	  \item System wyświetla komunikat informujący o miejscu oraz typie błędu w
	  wprowadzonych danych
	  \item Dane zostają poprawione
	  \item Kroki następne scenariusza głównego od 3 włącznie.
	\end{enumerate}
	
	Dodawanie pracownika - scenariusz alternatywny - pracownik już zarejestrowany
	\begin{enumerate}
	  \item Kroki 1-3 scenariusza głównego
	  \item System oświadcza, że wprowadzone dane odpowiadają istniejącemu już
	  pracownikowi - wskazując pokrywające się informacje
	  \item Po akceptacji, formularz jest odrzucany.
	\end{enumerate}
