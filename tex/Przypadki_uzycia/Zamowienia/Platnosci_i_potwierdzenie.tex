  \item Edycja formy płatności\\
  \begin{tabularx}{\linewidth}{c X}
  Aktor: & Pracownik \\
  Opis: & Pracownik ma możliwość zmiany początkowo wybranej formy płatności
  danego zamówienia. Odbywa się to na wniosek zamawiającego lub osoby
  odpowiedzialnej za zamówienia po stronie Sklepu.
  \end{tabularx}    
	\begin{enumerate}
	  \item Z listy zamówień pracownik wybiera jedno i wybiera opcję Zmiana formy
	  płatności
	  \item System prezentuje widok wyboru pomiędzy dostępnymi formami płatności
	  (specyfikacja w wymaganiach niefunkcjonalnych punkt \ref{itm:Platnosci})
	  \item Pracownik dokonuje wyboru formy oraz waluty.
	  \item System powiadamia klienta o zmianie formy płatności drogą elektroniczną.
	\end{enumerate}

  \item Wybór sposobu potwierdzenia zamówienia\\
  \begin{tabularx}{\linewidth}{c X}
  Aktor: & Pracownik \\
  Opis: & Możliwość zmiany sposobu udokumentowania przeprowadzonej transakcji
  (zazwyczaj będzie to faktura albo paragon). Powodem takich zmian mogą być
  nawet regulacje prawne.
  \end{tabularx}	
	\begin{enumerate}
	  \item Z listy zamówień pracownik wybiera jedno i wybiera opcję Zmiana
	  Potwierdzenia Transakcji
	  \item System prezentuje widok wyboru pomiędzy dostępnymi sposobami
	  potwierdzenia (udokumentowania) prowadzonej transakcji (wymagania
	  niefunkcyjne punkt \ref{itm:PotwierdzenieTransakcji})
	  \item Pracownik dokonuje wyboru oraz może uruchomić proces generacji
	  dokumentu.
	  \item W przypadku generacji dokumenty system wyświetla go pracownikowi.
	  \item Po akceptacji informacje o zmianie wraz z dokumentami wysyłane są drogą
	  elektroniczną do klienta.
	\end{enumerate}

  \item Generowanie faktury pro-forma dla danego zamówienia\\

  Opis słowny - niekiedy klient wymaga dokumentacji wystawionego zamówienia
  jeszcze zanim go odbierze. Uznaną prawnie formą takiej dokumentacji jest
  faktura pro-forma, która może być wygenerowana z danego zamówienia

  \begin{longtable}{|p{5cm}|p{7cm}|}
 	\hline
	\textbf{Aktor} & Pracownik \\
	\hline
	\textbf{Warunki początkowe} & Pracownik zalogowany, dokument generowany na
	wniosek klienta
	\\
	\hline
	\textbf{Opis przebiegu interakcji} & Wybór zarządzania zamówieniami,
	z listy zamówień zaznaczenie konkretnego, opcja generacji faktury
	\\
	\hline
	\textbf{Sytuacje wyjątkowe} & Brak żądanego zamówienia.
	\\
	\hline
	\textbf{Warunki końcowe} & Wytworzona faktura pro-forma - przesłana dla
	klienta.
	\\
	\hline
 \end{longtable}

  \begin{tabularx}{\linewidth}{c X}
  Aktor: & Pracownik \\
  Opis: & Możliwość utworzenie faktury pro-forma na podstawie danego zamówienia
  oraz przesłanie jej klientowi drogą elektroniczną lub wydruk.
  \end{tabularx}
	\begin{enumerate}
	  \item Z listy zamówień pracownik wybiera jedno i wybiera opcję Generuj
	  Pro-Forma
	  \item Dla wybranego zamówienia system generuje pełną fakturę po czym
	  prezentuje ją pracownikowi
	  \item Pracownik ma możliwość odrzucenia lub akceptacji dokumentu.
	  \item W przypadku akceptacji system wyświetla widok wyboru z opcjami wysyłki
	  do klienta.
	  \item Po wybraniu pożądanej opcji przez pracownika, system wysyła dokument do
	  klienta albo do drukarki.
	\end{enumerate}

	Generowanie faktury pro-forma - scenariusz alternatywny - brak danego
	zamówienia
	\begin{enumerate}
	  \item Krok 1 z scenariusza głównego.
	  \item System stwierdza iż poszukiwane (na wniosek klienta) zamówienie nie
	  istnieje i pyta o rozwiązanie tego problemu
	  \item Pracownik wybiera jedną z dostęnych opcji naprawy sytuacji
	  \item System generuje wiadomość do klienta z zaistniałą sytuacją
	\end{enumerate}
