  \item Prezentacja zamówień\\
  
  Opis słowny - poniższe przypadki opisują dostęp i zarządzanie zamównieniami. W
  tym celu wprowadzono panel z listą wszystkich zamówień, którą można przeglądać
  oraz sprawdzać szczegóły poszczególnych pozycji. Po zaznaczeniu konkretnego
  zamówienia jest możliwość wyboru wielu opcji, w tym modyfikacji.
  
  \begin{longtable}{|p{5cm}|p{7cm}|}
 	\hline
	\textbf{Aktor} & Pracownik \\
	\hline
	\textbf{Warunki początkowe} & Pracownik zalogowany, posiada uprawnienia do
	manipulacji zamówieniami
	\\
	\hline
	\textbf{Opis przebiegu interakcji} & Wybór panelu zamówień, aktywacja wybranych
	opcji
	\\
	\hline
	\textbf{Sytuacje wyjątkowe} & brak \\
	\hline
	\textbf{Warunki końcowe} & Prezentacja listy zamówień, możliwość ich edycji \\
	\hline
 \end{longtable}
  
  \begin{tabularx}{\linewidth}{ c X }
  Aktor: & Pracownik \\
  Opis: & Prezentacja panelu z listą wszystkich zamówień znajdujących się w~systemie
  oraz możliwościami kontroli i zarządzania nimi.\\
  \end{tabularx}
	\begin{enumerate}
	  \item Pracownik loguje się w Panelu Zarządzania
	  \item Wybiera Panel Zarządzania Zamówieniami
	  \item Wyświetlana jest lista zamówień z możliwością modyfikacji widoków
	  oraz panelem opcji (wszystkie opisane w poniższych przypadkach użycia)
	\end{enumerate}

  \item Edycja, modyfikacja zamówień\\
  \begin{tabularx}{\linewidth}{c X}
  Aktor: & Pracownik \\
  Opis: & Funkcjonalność umożliwia modyfikację produktów w zamówieniu oraz
  zmianę danych odbiorcy.
  \end{tabularx}
	\begin{enumerate}
	  \item Pracownik po autoryzacji w panelu sterowania systemu, przechodzi do
	  panelu Zarządzania Zamówieniami (patrz Zamowienia przypadek użycia 1)
	  \item Z wyświetlonej przez system listy zamówień, pracownik wybiera jeden
	  element
	  \item W celu edycji produktów:
		\begin{enumerate}
		  \item Wybiera opcję podglądu zawartości zamówienia
		  \item Z wyświetlonej listy zamówionych produktów zaznacza jedną pozycję
		  \item Wybiera opcję edycji
		  \item Otrzymuje informacje o konkretnym produkcie (jego ID, szczegółowy opis)
		  oraz zamówioną ilość oraz podsumowanie (cenę, informację o udzielonych rabatach na dany produkt)
		  \item Pole z ilością produktu umożliwia modyfikację – wystarczy wprowadzić
		  liczbę z zakresu od 1 do maksymalnej liczby aktualnie dostępnych produktów w
		  magazynie (0 nie wchodzi w zakres bo do tego służy funkcja usunięcia)
		\end{enumerate}
	  \item W celu usunięcia produktu:
		\begin{enumerate}
		  \item Wybiera opcję podglądu zawartości zamówienia
		  \item Z wyświetlonej listy zamówionych produktów zaznacza jedną pozycję
		  \item Wybiera opcję Usuń
		  \item System pyta o potwierdzenie i po akceptacji dokonuje wykluczenia
		  produktu z zamówienia oraz wysyła powiadomienie do Zamawiającego
		\end{enumerate}
	  \item W celu dodania produktu:
		\begin{enumerate}
		  \item Wybiera opcję podglądu zawartości zamówienia
		  \item Wybiera opcję Dodaj produkt
		  \item Otworzony zostaje system zakupowy\\ 
		  (przebieg wyboru produktu - opisany w przypadkach użycia odnoszących się do
		  Produktów)
		  \item Po wybraniu produktu system wyświetla informację o tym jakie zostaną
		  wprowadzone zmiany i czeka na akceptację
		  \item Po akceptacji, zamówienie zostaje zmodyfikowane (produkt dodany),
		  koszt zaktualizowany oraz system informuje odbiorcę zamówienia (klienta) o
		  zaszłych zmianach – za pomocą wiadomości email (z ewentualną poprawioną
		  fakturą pro-forma, jeśli była zaznaczona taka opcja) 
	  \end{enumerate} %koniec alternatywy Dodania produktu
	\end{enumerate} %koniec UC Edycja, modyfikacja zamówień
	
  \item Usunięcie zamówienia w całości\\
  \begin{tabularx}{\linewidth}{c X}
  Aktor: & Pracownik \\
  Opis: & Istnieje możliwość anulowania zamówienia – na życzenie klienta lub z
  powodów biznesowych sklepu.
  \end{tabularx}  
	\begin{enumerate}
	  \item Pracownik po autoryzacji w panelu sterowania systemu, przechodzi do
	  panelu Zarządzania Zamówieniami (patrz Zamówienia przypadek użycia 1)
	  \item Z wyświetlonej przez system listy zamówień, pracownik wybiera jeden
	  element i wybiera opcję Usuń zamówienie
	  \item System wyświetla ostrzeżenie (wraz ze szczegółową informacją o
	  zamówieniu) i pyta o potwierdzenie
	  \item Pracownik potwierdza chęć usunięcia danego zamówienia. Ma też możliwość
	  wpisania krótkiego uzasadnienia tej operacji
	  \item System dokonuje usunięcia oraz wysyła powiadomienie o anulowaniu
	  zamówienia do zamawiającego (drogą elektroniczną)
	\end{enumerate}

	