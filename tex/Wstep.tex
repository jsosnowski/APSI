\section{Wstęp}

Niniejszy dokument został utworzony w celu opisania i
zaprezentowania realizacji projektu systemu internetowego dla firmy Bike
Shop sp. z.o.o. Firma ta jest jednym z największych dostawców części
rowerowych na terenie Warszawy. Wraz ze wzrostem liczby osób
zainteresowanych korzystaniem z ruchu jednośladowego i przewidywanym
dalszym intensywnym rozwojem tego środka transportu pojawiła się
konieczność stworzenia portalu gromadzącego wszystkie zamówienia
składane przez klientów zarówno przez Internet jak i w oddziałach firmy
rozmieszczonych na terenie Warszawy i okolicznych miejscowości.
Znacząco ułatwi to zarządzanie systemem dystrybucji a także umożliwi
klientom śledzenie zmian w statusie poszukiwanej przez nich części,
rozszerzenie zamówienia lub złożenie zapytania co do konkretnego
modelu czy produktu danej firmy. Poza obsługą klienta ważnym
elementem systemu jest środowisko umożliwiające dodawanie informacji
o nowych produktach lub ich aktualizacja w bazie danych wszystkich
produktów.

Realizacja systemu została zlecona przez firmę Bike Shop sp. z.o.o.
(zwaną dalej Zleceniodawcą) firmie APSI Programmers, zwanej dalej
Zleceniobiorcą. Zleceniobiorca odpowiedzialny jest za stworzenie zarówno
części umożliwiającej składanie i obsługę zamówień klientów, jak i
część administracyjną, dostępną dla pracowników firmy. Ważnym
elementem całego systemu jest zewnętrzna baza danych, przechowująca
informacje o wszystkich dostępnych produktach, klientach oraz statusach
ich zamówień.
System powinien spełniać wszystkie wymagania biznesowe postawione przed
Zleceniobiorcą, które zostaną opisane w dalszej części dokumentu.