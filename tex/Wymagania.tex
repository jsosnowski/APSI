\newpage
\section{Wymagania}

% Dla wymagań funkcjonalnych i niefunkcjonalnych
% Globalnie zakładam, że zagnieżdżone listy będą numerowane cyframi
\setenumerate{label*=\arabic*.}

W tej sekcji znajduje się lista wymagań jakie spełniać powinien budowany system.
Podane są one z podziałem na dwie kategorie. Pierwsza to wymagania funkcjonalne
określające funkcjonalności systemu oraz sposoby ich użycia. Druga natomist to
wymagania niefunkcjonalne, które opisują ilościowe i jakościowe warunki
działania systemu.

\subsection{Wymagania funkcjonalne}
\subsubsection{Klient}

Wymagania funkcjonalne dotyczące zamówień realizowanych przez sklep:

\setenumerate{label*=\arabic*.}

\begin{enumerate}
  \item Dodanie nowego klienta
  \item Edycja danych klienta
  \begin{enumerate}
    \item Edycja adresu klienta
    \item Edycja adresu e-mail
  \end{enumerate}
  \item Edycja czułych danych klienta
  \begin{enumerate}
    \item Edycja hasła
    \item Edycja statusu (stały klient, nowy klient)
  \end{enumerate}
  \item Wyrejestrowanie się klienta
  \item Usunięcie klienta
\end{enumerate}

Opis przypadków użycia:
\subsubsection{Zamówienia}

Wymagania funkcjonalne dotyczące zamówień realizowanych przez sklep:

\setenumerate{label*=\arabic*.}

\begin{enumerate}
  \item Prezentacja zamówień
  \item Edycja, modyfikacja
  \begin{enumerate}
    \item Dodanie lub usunięcie produktu z zamówienia
    \item Zmiana ilości produktu
    \item Zmiana danych zamawiającego
  \end{enumerate}
  \item Usunięcie zamówienia w całości
  \item Wybór formy płatności
  \begin{enumerate}
    \item Płatność gotówką
    \begin{enumerate}
      \item Koszt w złotówkach
      \item Koszt w euro
      \item Koszt w wirtualnej walucie
    \end{enumerate}
    \item Płatność przelewem
    \item Płatność ratalna oparta o system szybkich pożyczek SuperBank
    \item Możliwość wpłaty zaliczki przed wysyłką
    \item Obniżenie kosztu o naliczone rabaty i zniżki
  \end{enumerate}
  \item Wybór sposobu potwierdzenia zamówienia (faktura, paragon)
  \item Generowanie faktury pro-forma dla danego zmówienia
  \item Zarządzanie terminem dostawy
  \item Ustawianie aktualnego stanu zamówienia.
\end{enumerate}
\subsubsection{Części}

Wymagania funkcjonalne dotyczące sprzedaży części rowerowych:

\begin{enumerate}
  \item Lista części
  \begin{enumerate}
	  \item Prezentacja listy dostępnych części
	  \item Wyszukiwanie części
  \end{enumerate}
  \item Dane części
  \begin{enumerate}
	  \item Dodanie/usunięcie nowego typu części do/z magazynu
	  \item Edycja danych części (kod, nazwa, opis, zdjęcie, cena jednostkowa)
	  \item Edycja ilości sztuk danego typu części aktualnie znajdujących się na magazynie
	  \item Możliwość włączenia/wyłączenia części do/z sprzedaży (ukrycie przed klientem)
  \end{enumerate}
  \item Dostawy
  \begin{enumerate}
    \item Generowanie zamówienia na dostawę części, których ilość w magazynie spadnie poniżej zadanego poziomu
    \item Wprowadzenie do systemu dostawy części do magazynu
  \end{enumerate}
  \item Ewidencja
  \begin{enumerate}
    \item Prezentacja stanu magazynu
    \item Prezentacja zmian stanu magazynu w zadanym okresie czasowym
    \item Prezentacja zmian ilości sztuk danej części w zadanym okresie czasowym
  \end{enumerate}
\end{enumerate}
\subsubsection{Pracownik}

Wymagania funkcjonalne dotyczące obsługi pracowników w systemie:

\begin{enumerate}
  \item Zatrudnienie nowego pracownika
  \item Zwolnienie pracownika
  \item Ustalanie urlopów
  \begin{enumerate}
    \item Zgłaszanie próśb o urlop.
    \item Zarządzanie urlopami.
      \begin{enumerate}
        \item Rozpatrywanie próśb o urlop.
        \item Automatyczna aktualizacja statusów urlopów przez system.
      \end{enumerate}
  \end{enumerate}
\end{enumerate}


\pagebreak

W celu zapewnienia jak najlepszej czytelności i jak największej modularności
rozwiązania zdecydowano się na podział całej funkcjonalności na 4 podstawowe
działy:
\begin{enumerate}
	\item Klienci
	\item Pracownicy
	\item Zamowienia
	\item Części
\end{enumerate}

Nie można oczywiście mówić o sztywnym podziale między modułami i zupełnym braku
powiązań pomiędzy nimi. Siła modelu obiektowego leży przede wszystkim w
komunikacji pomiędzy poszczególnymi obiektami, zatem powiązaniami pomiędzy
klasami są niezbędne w procesie prawidłowego funkcjonowania systemu (sklepu
internetowego).

W dalszej części rozdziału opisane zostaną poszczególne moduły - dla każdego z
nich przedstawiony zostanie odpowiedni wycinek diagramu klas, który zostanie
opisany w sposób bardziej szczegółowy, w każdym przypadku użycia zostaną także
zawarte diagramy sekwencji i scenariusze alternatywne
\subsubsection{Opis przypadków użycia - klient}

Opis przypadków użycia wyjaśniające funkcjonalności związane z zarządzaniem
klientami:

\begin{enumerate}
  \item Rejestracja klienta \\
  \begin{tabularx}{\linewidth}{ c X }
  Aktor: & Klient \\
  Opis: & Możliwość rejestracji nowego klienta.\\
  \end{tabularx}
   \begin{enumerate}
    \item Klient uruchamia stronę internetową sklepu i wybiera opcję rejestracji
    \item Klient wstawia swoje dane osobowe i wybiera domyślny model płatności
    (kartą, za pobraniem itp.)
    \item System sprawdza wstawione dane (takie same hasła, czy istnieje już
    zarejestrowany w systemie użytkownik, czy istnieje podany adres e-mail itp.)
    \item System wysyła e-mail powitalny na adres podany przez klienta
    \item W ciągu określonego, zdefiniowanego czasu klient wybiera przesłany w
    e-mailu link, stają się pełnoprawnym użytkownikiem sklepu
  \end{enumerate}
  \item Złożenie zamówienia \\
  \begin{tabularx}{\linewidth}{ c X }
  Aktor: & Klient \\
  Opis: & Przedstawienie sposobu złożenia zamówienia.\\
  \end{tabularx}
  \begin{enumerate}
    \item Klient uruchamia stronę internetową sklepu i wyszukuje interesujące go
    produkty
    \item W momencie znalezienia pasującego produktu użytkownik wybiera opcję
    dodania do koszyka
    \item Po zakończeniu wyszukiwania użytkownik wybiera opcję przejścia do kasy
    \item System sprawdza, czy użytkownik jest zalogowany. Jeśli nie, procesuje
    przypadek użycia Logowanie do Systemu
    \item System sprawdza, czy użytkownik jest stałym klientem. Jeśli tak,
    dolicza rabat do ustalonej ceny (do sumy cen poszczególnych produktów)
    \item Użytkownik wybiera sposób płatności
    \item System dodaje do wcześniej ustalonej ceny koszty wynikające ze sposobu
    płatności
    \item Użytkownik wybiera sposób dostawy (poczta, kurier, odbiór osobisty
    itp.)
    \item System dodaje do ceny koszty wynikające ze sposobu dostawy
    \item Użytkownik, po sprawdzeniu wszystkich danych, decyduje się na złożenie
    zamówienia - po tym momencie nie może już ono być cofnięte
    \item System wysyła do użytkownika e-mail potwierdzający wraz z przewidywaną
    datą realizacji zamówienia
  \end{enumerate} 
  \item Edycja danych klienta \\
  \begin{tabularx}{\linewidth}{ c X }
  Aktor: & Klient \\
  Opis: & Możliwość zmiany, uzupełnienia danych osobowych klienta.\\
  \end{tabularx}
  \begin{enumerate}
    \item Klient uruchamia witrynę internetową sklepu
    \item Klient loguje się do systemu (tylko osoba zalogowana może zmieniać
    swoje dane)
    \item Klient edytuje wybrane pozycje ze swojego opisu (adres, numer
    telefonu itp.)
    \item W przypadku zmiany hasła klient proszony jest o podanie starego jak i
    nowego (dwukrotnie) hasła
    \item Klient zatwierdza wprowadzone zmiany
    \item System wysyła na podany przez użytkownika adres e-mail (nowy, jeśli
    to adres e-mail był jedną ze zmienianych wartości) informację o zmianie.
  \end{enumerate}
  \item Wyrejestrowanie się klienta \\
  \begin{tabularx}{\linewidth}{ c X }
  Aktor: & Klient \\
  Opis: & Klient ma możliwość w każdym momencie usunąć swoje konto z systemu.\\
  \end{tabularx}
  \begin{enumerate}
    \item Klient uruchamia witrynę internetową i loguje się na swoje konto
    (przypadek użycia Logowanie Do Systemu)
    \item Klient wybiera opcję usunięcia danych
    \item System sprawdza, czy istnieją niezrealizowane (oczekujące) zamówienia.
    Jeśli tak, wyświetla się alert z informacją, czy dane zamówienie zostało już
    wcześniej opłacone
    \item Jeśli istniały już zamówienia, które zostały opłacone a nie zostały
    jeszcze zrealizowane, system zleca odesłanie określonej kwoty pieniężnej z
    powrotem na konto użytkownika (z pominięciem kosztów obsługi)
    \item Klient zostaje poproszony o podanie przyczyn swojej decyzji -
    wypełnianie jest nieobowiązkowe
    \item Dane przechowywane są przez następne 7 dni (wymaganie prawne). W tym
    czasie klient może ponownie zarejestrować się w systemie bez utraty
    poprzednich danych
    \item W przypadku braku ponownej rejestracji dane zostają na stałe usunięte
    z firmowej bazy danych
  \end{enumerate}
  \item Usunięcie klienta \\
  \begin{tabularx}{\linewidth}{ c X }
  Aktor: & Pracownik \\
  Opis: & Klienta można usunąć administracyjnie na przykład z powodów
  naruszenia regulaminu.\\
  \end{tabularx}
  \begin{enumerate}
    \item Pracownik sklepu wyszukuje klienta o konkretnym imieniu i nazwisku
    (lub według innych kryteriów)
    \item Pracownik wybiera opcję usunięcia klienta. 
    \item Pracownik wpisuje powód, dla którego usuwa użytkownika (informacja ta
    będzie przesłana do klienta w wiadomości e-mail)
    \item Pracownik wypełnia dane dotyczące kwestii niezrealizowanych zamówień i
    nieotrzymanych płatności
    \item Obie informacji (z poprzednich 2 kroków) są przekazywane na podany
    przez użytkownika adres e-mail
    \item Dane są przechowywane przez następne 30 dni - w tym czasie użytkownik
    może złożyć reklamację i ewentualnie odzyskać dostęp do konta
    \item Po 30 dniach, jeśli nie zostanie pozytywnie rozpatrzona prośba o
    przywrócenie konta, dane są na stałe usuwane z bazy danych
  \end{enumerate}
\end{enumerate}
\subsubsection{Opis przypadków użycia - zamówienia}

Przypadki użycia wyjaśniające funkcjonalności systemu związane z zarządzaniem
zamówieniami.

\begin{enumerate}
  \item Prezentacja zamówień\\
  \begin{tabularx}{\linewidth}{ c X }
  Aktor: & Pracownik \\
  Opis: & Prezentacja panelu z listą wszystkich zamówień znajdujących się w~systemie
  oraz możliwościami kontroli i zarządzania nimi.\\
  \end{tabularx}
	\begin{enumerate}
	  \item Pracownik loguje się w Panelu Zarządzania
	  \item Wybiera Panel Zarządzania Zamówieniami
	  \item Wyświetlana jest lista zamówień z możliwością modyfikacji widoków
	  oraz panelem opcji (wszystkie opisane w poniższych przypadkach użycia)
	\end{enumerate}

  \item Edycja, modyfikacja zamówień\\
  \begin{tabularx}{\linewidth}{c X}
  Aktor: & Pracownik \\
  Opis: & Funkcjonalność umożliwia modyfikację produktów w zamówieniu oraz
  zmianę danych odbiorcy.
  \end{tabularx}
	\begin{enumerate}
	  \item Pracownik po autoryzacji w panelu sterowania systemu, przechodzi do
	  panelu Zarządzania Zamówieniami (patrz Zamowienia przypadek użycia 1)
	  \item Z wyświetlonej przez system listy zamówień, pracownik wybiera jeden
	  element
	  \item W celu edycji produktów:
		\begin{enumerate}
		  \item Wybiera opcję podglądu zawartości zamówienia
		  \item Z wyświetlonej listy zamówionych produktów zaznacza jedną pozycję
		  \item Wybiera opcję edycji
		  \item Otrzymuje informacje o konkretnym produkcie (jego ID, szczegółowy opis)
		  oraz zamówioną ilość oraz podsumowanie (cenę, informację o udzielonych rabatach na dany produkt)
		  \item Pole z ilością produktu umożliwia modyfikację – wystarczy wprowadzić
		  liczbę z zakresu od 1 do maksymalnej liczby aktualnie dostępnych produktów w
		  magazynie (0 nie wchodzi w zakres bo do tego służy funkcja usunięcia)
		\end{enumerate}
	  \item W celu usunięcia produktu:
		\begin{enumerate}
		  \item Wybiera opcję podglądu zawartości zamówienia
		  \item Z wyświetlonej listy zamówionych produktów zaznacza jedną pozycję
		  \item Wybiera opcję Usuń
		  \item System pyta o potwierdzenie i po akceptacji dokonuje wykluczenia
		  produktu z zamówienia oraz wysyła powiadomienie do Zamawiającego
		\end{enumerate}
	  \item W celu dodania produktu:
		\begin{enumerate}
		  \item Wybiera opcję podglądu zawartości zamówienia
		  \item Wybiera opcję Dodaj produkt
		  \item Otworzony zostaje system zakupowy\\ 
		  (przebieg wyboru produktu - opisany w przypadkach użycia odnoszących się do
		  Produktów)
		  \item Po wybraniu produktu system wyświetla informację o tym jakie zostaną
		  wprowadzone zmiany i czeka na akceptację
		  \item Po akceptacji, zamówienie zostaje zmodyfikowane (produkt dodany),
		  koszt zaktualizowany oraz system informuje odbiorcę zamówienia (klienta) o
		  zaszłych zmianach – za pomocą wiadomości email (z ewentualną poprawioną
		  fakturą pro-forma, jeśli była zaznaczona taka opcja) 
	  \end{enumerate} %koniec alternatywy Dodania produktu
	\end{enumerate} %koniec UC Edycja, modyfikacja zamówień
  
  \item Zmiana danych zamawiającego\\
  \begin{tabularx}{\linewidth}{c X}
  Aktor: & Pracownik \\
  Opis: & Można zmienić dane odbiorcy na potrzeby danego zamówienia (zmiana
  danych tylko w ramach konkretnej faktury). Dotyczy to w szczególności adresu i
  danych osobowych osoby odpowiedzialnej za zamówienie.
  \end{tabularx}  
	\begin{enumerate}
	  \item Pracownik po autoryzacji w panelu sterowania systemu, przechodzi do
	  panelu Zarządzania Zamówieniami (patrz Zamówienia przypadek użycia 1)
	  \item Z wyświetlonej przez system listy zamówień, pracownik wybiera jeden
	  element i wybiera opcję Zmień Dane Odbiorcy
	  \item System prezentuje aktualnie dane odbiorcy (mogą to być aktualne dane
	  klienta, albo już wcześniej modyfikowane dane osobowe wprowadzone specjalnie w
	  ramach tego zamówienia)
	  \item Pracownik modyfikuje wybraną przez siebie składową danych (wszystkie
	  elementy pozwalają na edycję) i zatwierdza wprowadzone zmiany
	  \item System wyświetla zapytanie o potwierdzenie zmian i po jego akceptacji
	  wysyła powiadomienie do klienta o zaszłych zmianach – wiadomość drogą
	  elektroniczną
	\end{enumerate}

  \item Usunięcie zamówienia w całości\\
  \begin{tabularx}{\linewidth}{c X}
  Aktor: & Pracownik \\
  Opis: & Istnieje możliwość anulowania zamówienia – na życzenie klienta lub z
  powodów biznesowych sklepu.
  \end{tabularx}  
	\begin{enumerate}
	  \item Pracownik po autoryzacji w panelu sterowania systemu, przechodzi do
	  panelu Zarządzania Zamówieniami (patrz Zamówienia przypadek użycia 1)
	  \item Z wyświetlonej przez system listy zamówień, pracownik wybiera jeden
	  element i wybiera opcję Usuń zamówienie
	  \item System wyświetla ostrzeżenie (wraz ze szczegółową informacją o
	  zamówieniu) i pyta o potwierdzenie
	  \item Pracownik potwierdza chęć usunięcia danego zamówienia. Ma też możliwość
	  wpisania krótkiego uzasadnienia tej operacji
	  \item System dokonuje usunięcia oraz wysyła powiadomienie o anulowaniu
	  zamówienia do zamawiającego (drogą elektroniczną)
	\end{enumerate}

  \item Edycja formy płatności\\
  \begin{tabularx}{\linewidth}{c X}
  Aktor: & Pracownik \\
  Opis: & Pracownik ma możliwość zmiany początkowo wybranej formy płatności
  danego zamówienia. Odbywa się to na wniosek zamawiającego lub osoby
  odpowiedzialnej za zamówienia po stronie Sklepu.
  \end{tabularx}    
	\begin{enumerate}
	  \item Z listy zamówień pracownik wybiera jedno i wybiera opcję Zmiana formy
	  płatności
	  \item System prezentuje widok wyboru pomiędzy dostępnymi formami płatności
	  (specyfikacja w wymaganiach niefunkcjonalnych punkt \ref{itm:Platnosci})
	  \item Pracownik dokonuje wyboru formy oraz waluty.
	  \item System powiadamia klienta o zmianie formy płatności drogą elektroniczną.
	\end{enumerate}

  \item Wybór sposobu potwierdzenia zamówienia\\
  \begin{tabularx}{\linewidth}{c X}
  Aktor: & Pracownik \\
  Opis: & Możliwość zmiany sposobu udokumentowania przeprowadzonej transakcji
  (zazwyczaj będzie to faktura albo paragon). Powodem takich zmian mogą być
  nawet regulacje prawne.
  \end{tabularx}	
	\begin{enumerate}
	  \item Z listy zamówień pracownik wybiera jedno i wybiera opcję Zmiana
	  Potwierdzenia Transakcji
	  \item System prezentuje widok wyboru pomiędzy dostępnymi sposobami
	  potwierdzenia (udokumentowania) prowadzonej transakcji (wymagania
	  niefunkcyjne punkt \ref{itm:PotwierdzenieTransakcji})
	  \item Pracownik dokonuje wyboru oraz może uruchomić proces generacji
	  dokumentu.
	  \item W przypadku generacji dokumenty system wyświetla go pracownikowi.
	  \item Po akceptacji informacje o zmianie wraz z dokumentami wysyłane są drogą
	  elektroniczną do klienta.
	\end{enumerate}

  \item Generowanie faktury pro-forma dla danego zmówienia\\
  \begin{tabularx}{\linewidth}{c X}
  Aktor: & Pracownik \\
  Opis: & Możliwość utworzenie faktury pro-forma na podstawie danego zamówienia
  oraz przesłanie jej klientowi drogą elektroniczną lub wydruk.
  \end{tabularx}
	\begin{enumerate}
	  \item Z listy zamówień pracownik wybiera jedno i wybiera opcję Generuj
	  Pro-Forma
	  \item Dla wybranego zamówienia system generuje pełną fakturę po czym
	  prezentuje ją pracownikowi
	  \item Pracownik ma możliwość odrzucenia lub akceptacji dokumentu.
	  \item W przypadku akceptacji system wyświetla widok wyboru z opcjami wysyłki
	  do klienta.
	  \item Po wybraniu pożądanej opcji przez pracownika, system wysyła dokument do
	  klienta albo do drukarki.
	\end{enumerate}

  \item Zarządzanie dostawą\\
  \begin{tabularx}{\linewidth}{c X}
  Aktor: & Pracownik \\
  Opis: & Termin realizacji zamówienia oraz sposób dostawy mogą być
  modyfikowany dowolnie w zależności od możliwości biznesowych Sklepu i
  aktualnego stanu zamówienia.
  \end{tabularx}
	\begin{enumerate}
	  \item Z listy zamówień użytkownik wybiera jedno i wybiera opcję Edycji
	  Dostawy
	  \item System prezentuje informacje o wybranym sposobie i terminie dostawy
	  \item Użytkownik wybiera opcję Zmiany daty realizacji
	  \item System prezentuje widok kalendarza z zaznaczoną dotychczasową datą
	  realizacji.
	  \item Użytkownik przesuwa datę realizacji projektu i ma możliwość podania
	  wiadomości wyjaśniającej modyfikację.
	  \item Użytkownik wybiera opcję Zmiany sposobu dostawy
	  \item System wyświetla wszystkie aktualnie dostępne opcje razem ze
	  szczegółami (cena, średni czas)
	  \item Użytkownik dokonuje wyboru środka transportu i zatwierdza zmiany
	  \item System aktualizuje koszt całego zamówienia uwzględniając kwotę
	  transportu oraz wysyła powiadomienie o zmianach (termin lub/i sposób dostawy)
	  do klienta wraz z informacją wyjaśniającą wpisaną przez pracownika.
	\end{enumerate}

  \item Ustawianie aktualnego stanu zamówienia\\
  \begin{tabularx}{\linewidth}{c X}
  Aktor: & Pracownik \\
  Opis: & Zamówienie może znajdować się w pewnych stanach realizacji (np. w
  przygotowaniu, w realizacji, wysłane - konkretne stany określają wymagania
  niefunkcjonalne). Istnieje możliwość zmiany aktualnego stanu zamówienia.
  \end{tabularx}
	\begin{enumerate}
	  \item Z listy zamówień pracownik wybiera jedno i wybiera opcję Zmień Stan
	  \item System prezentuje widok z dostępnymi stanami dla danego zamówienia
	  \item Pracownik dokonuje wyboru i zatwierdza zmiany.
	  \item Jeśli pracownik wybiera opcję Powiadom, to system powiadamia klienta o
	  zmianie stanu jaka nastąpiła i przesyła krótkie wyjaśnienie.
	\end{enumerate}
	 
\end{enumerate}

\subsubsection{Opis przypadków użycia - części}

Poniżej przedstawiono przypadki użycia związane z zarządzaniem częściami. W większości przypadków aktorem jest pracownik sklepu uprawniony do zarządzania częściami. Wyjątkiem jest lista części dostępnych do kupienia, którą może także wyświetlić klient.

\begin{figure}[h!]
    \includegraphics[width=\textwidth,
    height=0.5\textheight]{graphics/UseCase/Czesci/UseCaseDiagram.png}
  \caption{Diagram przypadków użycia związanych z zarządzaniem częściami}
\end{figure}

\begin{figure}[h!]
    \includegraphics[width=\textwidth,
    height=0.5\textheight]{graphics/UseCase/Czesci/DosatwyUseCase.png}
  \caption{Diagram przypadków użycia związanych z zarządzaniem dostawami}
\end{figure}

\newpage
\begin{enumerate}
   \item Wyświetlanie listy części - scenariusz główny \\
 
 Opis słowny - jest to podstawowy przypadek użycia jeśli chodzi o zarządzanie częściami, gdyż zapewnia on funkcjonalność prezentacji wszystkich dostępnych w sklepie części. Jest to funkcjonalność używana zarówno przez klientów sklepu (w celu przeglądania asortymentu sklepu), jak i przez pracowników (w celu zarządzania częściami). W przypadku gdy użytkownik posiada odpowiednie uprawnienia (pracownik magazynu), lista części udostępnia także funkcjonalności zarządzania częściami (modyfikacja, usuwanie, itd. - dokładnie opisane poniżej).
 
 \begin{longtable}{|p{5cm}|p{7cm}|}
 	\hline
	\textbf{Aktor} & Klient lub pracownik \\
	\hline
	\textbf{Warunki początkowe} & Brak \\
	\hline
	\textbf{Opis przebiegu interakcji} & Wybór prezentacji listy części na stronie głównej sklepu \\
	\hline
	\textbf{Sytuacje wyjątkowe} & Ustawienie kryteriów filtrowania, posiadanie przez użytkownika specjalnych uprawnień do zarządania częściami \\
	\hline
	\textbf{Warunki końcowe} & Wyświetlenie listy dostępnych części \\
	\hline
 \end{longtable}
 
  \item Wyświetlanie listy części - scenariusz główny \label{lista-czesci} \\
  \begin{tabularx}{\linewidth}{ c X}
  Aktor: & Klient lub pracownik \\
  \end{tabularx}
   \begin{enumerate}
    \item Użytkownik (nie posiadający specjalnych uprawnień) uruchamia stronę internetową sklepu i wybiera opcję wyświetlenia listy części.
    \item System prezentuje listę dostępnych dla użytkownika części (możliwych do kupienia), posortowaną alfabetycznie, z podziałem wyników na strony (10 wartości na stronie, z możliwością zmiany tej liczby przez użytkownika na 25, 50 i 100).
  \end{enumerate}
  
  \item Wyświetlanie listy części - scenariusz alternatywny - ustawiono kryteria filtrowania \\
  \begin{tabularx}{\linewidth}{ c X}
  Aktor: & Klient lub pracownik \\
  \end{tabularx}
   \begin{enumerate}
     \item Krok 1 scenariusza głównego.
     \item Użytkownik ustawia żądane przez siebie kryteria filtrowania. Filtrować można po kodzie części, jej nazwie, opisie i cenie.
     \item System prezentuje odfiltrowaną listę części. Sposób prezentacji taki sam jak w scenariuszu głównym, kroku 2.
   \end{enumerate} 
   
   \item Wyświetlanie listy części - scenariusz alternatywny - użytkownik posiada specjalne uprawnienia do zarządzania częściami \\
   \begin{tabularx}{\linewidth}{ c X}
	Aktor: & Pracownik \\
  	\end{tabularx}   
  	\begin{enumerate}
     \item Krok 1 scenariusza głównego.
  	  \item System prezentuje listę części w taki sam sposób jak dla kroku 2 scenariusza głównego, ale dodatkowo przy każdym elemencie listy dostępne są przyciski, umożliwiające pracownikowi edycję danych części lub usunięcie jej z systemu. Na tym ekranie widoczny jest także przycisk umożliwiający dodanie nowego typu części do systemu. Działanie tych przycisków opisane jest w kolejnych przypadkach użycia. Dodatkowo na liście widoczne są także części ukryte (niewidoczne dla kupujących).
  	\end{enumerate}	
  	
\begin{figure}[h!]
    \includegraphics[width=\textwidth,
    height=0.5\textheight]{graphics/UseCase/Czesci/ListaCzesciSD.png}
  \caption{Diagram sekwencji dla przypadku użycia Lista części - scenariusz główny}
\end{figure}
   \item Dodanie części - scenariusz główny \\
 
 Opis słowny - ten przypadek użycia opisuje funkcjonalność dodawania nowych typów części, czyli takich które nie są jeszcze zarejestrowane w systemie. Obejmuje to sytuację, gdy np. sklep decyduje się na wprowadzenie nowego asortymentu do sprzedaży.
 
 \begin{longtable}{|p{5cm}|p{7cm}|}
 	\hline
	\textbf{Aktor} & Pracownik \\
	\hline
	\textbf{Warunki początkowe} & Posiadanie konta z uprawnieniami umożliwiającymi zarządzanie częściami, zalogowanie się do systemu \\
	\hline
	\textbf{Opis przebiegu interakcji} & Wybór prezentacji listy części na stronie głównej sklepu, wybranie opcji dodania nowej części, uzupełnienie danych, zatwierdzenie operacji \\
	\hline
	\textbf{Sytuacje wyjątkowe} & Dodawana część już istnieje, podanie błędnych danych nowej części \\
	\hline
	\textbf{Warunki końcowe} & Dodanie do systemu nowego typu części \\
	\hline
 \end{longtable}
 
  \item Dodanie części - scenariusz główny \\
  \begin{tabularx}{\linewidth}{ c X}
  Aktor: & Pracownik \\
  \end{tabularx}
   \begin{enumerate}
    \item Scenariusz ``Wyświetlanie listy części - scenariusz alternatywny - użytkownik posiada specjalne uprawnienia do zarządzania częściami''. \label{dodanie-czesci-poczatek}
    \item Pracownik wybiera opcję dodania nowej części.
    \item System prezentuje pracownikowi formatkę dodania nowej części, z możliwością wypełnienia następujących atrybutów części:
    \begin{enumerate}
      \item Kod części (generowany automatycznie, z możliwością edycji przez pracownika)
      \item Nazwa części
      \item Opis części
      \item Zdjęcie części (w jedym z popularnych formatów graficznych, takich jak JPG, PNG czy GIF)
      \item Cena jednostkowa części
      \item Widoczność części dla klientów (czy klienci będą mogli zobaczyć część na liście części możliwych do kupienia)
      \item Minimalna liczba sztuk w magazynie
    \end{enumerate}
    \item Pracownik uzupełnia wymagane pola i zatwierdza operację. \label{dodanie-czesci-zatwierdzenie}
    \item System dodaje część do bazy danych i informuje użytkownika o zakończeniu operacji.
  \end{enumerate}
  
  \item Dodanie części - scenariusz alternatywny - dodawana część już istnieje \\
  \begin{tabularx}{\linewidth}{ c X}
  Aktor: & Pracownik \\
  \end{tabularx}
   \begin{enumerate}
    \item Kroki \ref{dodanie-czesci-poczatek} do \ref{dodanie-czesci-zatwierdzenie} scenariusza głównego.
    \item System sprawdza czy istnieje już część o podanym kodzie. Jeśli tak, wyświetla komunikat o błędzie i anuluje operację. Jeśli nie, to powrót do scenariusza głównego.
  \end{enumerate}
  
  \item Dodanie części - scenariusz alternatywny - podanie błędnych danych nowej części \\
  \begin{tabularx}{\linewidth}{ c X}
  Aktor: & Pracownik \\
  \end{tabularx}
   \begin{enumerate}
    \item Kroki \ref{dodanie-czesci-poczatek} do \ref{dodanie-czesci-zatwierdzenie} scenariusza głównego.
    \item System sprawdza czy podane przez użytkownika dane są poprawne, czyli:
    \begin{enumerate}
      \item Czy wszystkie pola oprócz opisu i zdjęcia są wypełnione
      \item Czy podana cena nie jest ujemna
      \item Czy podana minimalna liczba sztuk na magazynie nie jest ujemna
    \end{enumerate}
    \item W przypadku niepoprawności wprowadzonych danych, system wyświetla stosowny komunikat błędu i anuluje operację. W przeciwnym przypadku powrót do scenariusza głównego.
  \end{enumerate}
  	
\begin{figure}[h!]
    \includegraphics[width=\textwidth,
    height=0.7\textheight]{graphics/UseCase/Czesci/DodanieCzesciSD.png}
  \caption{Diagram sekwencji dla przypadku użycia Dodanie części - scenariusz główny}
\end{figure}
   \item Usunięcie części - scenariusz główny \\
 
 Opis słowny - ten przypadek użycia opisuje funkcjonalność usuwania istniejących typów części. Obejmuje to sytuację, gdy np. sklep decyduje się na usunięcie starego asortymentu ze sprzedaży.
 
 \begin{longtable}{|p{5cm}|p{7cm}|}
 	\hline
	\textbf{Aktor} & Pracownik \\
	\hline
	\textbf{Warunki początkowe} & Posiadanie konta z uprawnieniami umożliwiającymi zarządzanie częściami, zalogowanie się do systemu \\
	\hline
	\textbf{Opis przebiegu interakcji} & Wybór prezentacji listy części na stronie głównej sklepu, wybranie opcji usunięcia konkretnej części, zatwierdzenie operacji \\
	\hline
	\textbf{Sytuacje wyjątkowe} & Istnieje niezerowa liczba części tego typu w magazynie \\
	\hline
	\textbf{Warunki końcowe} & Usunięcie z systemu wybranego typu części \\
	\hline
 \end{longtable}
 
  \item Usunięcie części - scenariusz główny \\
  \begin{tabularx}{\linewidth}{ c X}
  Aktor: & Pracownik \\
  \end{tabularx}
   \begin{enumerate}
    \item Scenariusz ``Wyświetlanie listy części - scenariusz alternatywny - użytkownik posiada specjalne uprawnienia do zarządzania częściami''. \label{usuniecie-czesci-poczatek}
    \item Pracownik wyszukuje na liście część którą chce usunąć i wybiera przycisk usunięcia. \label{usuniecie-czesci-wybor}
    \item System prosi użytkonika o potwierdzenie operacji.
    \item Pracownik zatwierdza operację.
    \item System usuwa część z bazy danych i informuje użytkownika o zakończeniu operacji.
  \end{enumerate}
  
  \item Dodanie części - scenariusz alternatywny - istnieje niezerowa liczba części tego typu w magazynie \\
  \begin{tabularx}{\linewidth}{ c X}
  Aktor: & Pracownik \\
  \end{tabularx}
   \begin{enumerate}
    \item Kroki \ref{usuniecie-czesci-poczatek} do \ref{usuniecie-czesci-wybor} scenariusza głównego.
    \item System sprawdza czy w magazynie znajdują się części danego typu. Jeśli tak (na magazynie znajduje się co najmniej 1 sztuka danego typu części), informuje o tym pracownika w formie ostrzeżenia.
    \item Powrót do scenariusza głównego.
  \end{enumerate}
  	
\begin{figure}[h!]
    \includegraphics[width=\textwidth,
    height=0.7\textheight]{graphics/UseCase/Czesci/UsuniecieCzesciSD.png}
  \caption{Diagram sekwencji dla przypadku użycia Usunięcie części - scenariusz główny}
\end{figure}
   \item Modyfikacja danych części - scenariusz główny \\
 
 Opis słowny - ten przypadek użycia opisuje funkcjonalność modyfikacji danych istniejących typów części. Obejmuje to sytuację, gdy np. cena danej części musi zostać zmieniona.
 
 \begin{longtable}{|p{5cm}|p{7cm}|}
 	\hline
	\textbf{Aktor} & Pracownik \\
	\hline
	\textbf{Warunki początkowe} & Posiadanie konta z uprawnieniami umożliwiającymi zarządzanie częściami, zalogowanie się do systemu \\
	\hline
	\textbf{Opis przebiegu interakcji} & Wybór prezentacji listy części na stronie głównej sklepu, wybranie opcji modyfikacji konkretnej części, wprowadzenie danych, zatwierdzenie operacji \\
	\hline
	\textbf{Sytuacje wyjątkowe} & Podanie błędnych danych przy modyfikacji części \\
	\hline
	\textbf{Warunki końcowe} & Modyfikacja danych wybranej części \\
	\hline
 \end{longtable}
 
  \item Modyfikacja danych części - scenariusz główny \\
  \begin{tabularx}{\linewidth}{ c X}
  Aktor: & Pracownik \\
  \end{tabularx}
   \begin{enumerate}
    \item Scenariusz ``Wyświetlanie listy części - scenariusz alternatywny - użytkownik posiada specjalne uprawnienia do zarządzania częściami''. \label{modyfikacja-czesci-poczatek}
    \item Pracownik wyszukuje na liście część którą chce zmodyfikować i wybiera przycisk modyfikacji.
    \item System prezentuje pracownikowi formatkę taką jak dla dodania nowej części, ale wypełnioną danymi modyfikowanej części, z możliwością ich edycji. Dodatkowo, istnieje możliwość edycji liczby części wybranego typu, znajdujących się aktualnie w magazynie.
    \item Pracownik modyfikuje wybrane pola i zatwierdza operację. \label{modyfikacja-czesci-zatwierdzenie}
    \item System modyfikuje część w bazie danych i informuje użytkownika o zakończeniu operacji.
  \end{enumerate}
  
  \item Modyfikacja danych części - scenariusz alternatywny - podanie błędnych danych przy modyfikacji części \\
  \begin{tabularx}{\linewidth}{ c X}
  Aktor: & Pracownik \\
  \end{tabularx}
   \begin{enumerate}
    \item Kroki \ref{modyfikacja-czesci-poczatek} do \ref{modyfikacja-czesci-zatwierdzenie} scenariusza głównego.
    \item System sprawdza czy zmodyfikowane przez użytkownika dane są poprawne, czyli:
    \begin{enumerate}
      \item Czy wszystkie pola oprócz opisu i zdjęcia są wypełnione
      \item Czy podana cena nie jest ujemna
      \item Czy podana minimalna liczba sztuk na magazynie nie jest ujemna
      \item Czy podana aktualna liczba sztuk na magazynie nie jest ujemna
    \end{enumerate}
    \item W przypadku niepoprawności wprowadzonych danych, system wyświetla stosowny komunikat błędu i anuluje operację. W przeciwnym przypadku powrót do scenariusza głównego.
  \end{enumerate}
  	
\begin{figure}[h!]
    \includegraphics[width=\textwidth,
    height=0.7\textheight]{graphics/UseCase/Czesci/ModyfikacjaCzesciSD.png}
  \caption{Diagram sekwencji dla przypadku użycia Modyfikacja części - scenariusz główny}
\end{figure}
   \item Generowanie zamówienia na dostawę części - scenariusz główny \\
 
 Opis słowny - ten przypadek użycia opisuje funkcjonalność generowania zamówienia na dostawę części, których ilość w magazynie spadnie poniżej zadanego poziomu, ustawianego oddzielnie dla każdej części. Jest to funkcjonalność bardzo ułatwiająca pracę pracownikom magazynu, którzy dbają o zaopatrzenie sklepu, ponieważ po ustawieniu minimalnej liczby sztuk towaru, system sam dba o to, żeby poziom ten zawsze był utrzymany, zwalniając z tego obowiązku pracowników. Skutkuje to także mniejszą liczbą pomyłek przy zamawianiu dostaw.
 
 \begin{longtable}{|p{5cm}|p{7cm}|}
 	\hline
	\textbf{Aktor} & Pracownik \\
	\hline
	\textbf{Warunki początkowe} & Posiadanie konta z uprawnieniami umożliwiającymi zarządzanie częściami, zalogowanie się do systemu \\
	\hline
	\textbf{Opis przebiegu interakcji} & Wybór opcji zarządzania dostawami na stronie głównej sklepu, wybranie opcji generowania zamówienia na dostawę, wprowadzenie danych, zatwierdzenie operacji \\
	\hline
	\textbf{Sytuacje wyjątkowe} & Brak \\
	\hline
	\textbf{Warunki końcowe} & Utworzenie dokumentu, zawierającego listę części które należy zamówić przy najbliższej dostawie \\
	\hline
 \end{longtable}
 
\begin{figure}[h!]
    \includegraphics[width=\textwidth,
    height=0.7\textheight]{graphics/UseCase/Czesci/GenerowanieZamowieniaSD.png}
  \caption{Diagram sekwencji dla przypadku użycia Generowanie zamówienia na dostawę części - scenariusz główny}
\end{figure}
  \item Generowanie zamówienia na dostawę części - scenariusz główny \\
  \begin{tabularx}{\linewidth}{ c X}
  Aktor: & Pracownik \\
  \end{tabularx}
   \begin{enumerate}
    \item Pracownik otwiera stronę internetową sklepu, loguje się na swoje konto i wybiera opcję zarządzania dostawami.
    \item System prezentuje widok generowania zamówienia na dostawę, umożliwiający:
    \begin{enumerate}
      \item Dodanie typu części do zamówienia.
      \item Usunięcie wcześniej wprowadzonego typu części z zamówienia.
      \item Prezentację listy już wprowadzonych części.
    \end{enumerate}
    \item System początkowo wypełnia listę tymi częściami, których ilość w magazynie spadła poniżej zadanego minimalnego poziomu. Części dodawane są w minimalnej ilości wystarczającej do tego, aby poziom ten został osiągnięty.
    \item Pracownik dodaje nowe części według następującego schematu:
    \begin{enumerate}
      \item Pracownik wybiera przycisk umożliwiający dodanie typu części do zamówienia.
      \item System prezentuje listę części z możliwością wyszukiwania tak jak w przypadku użycia ``Wyświetlanie listy części''.
      \item Pracownik wybiera żądany typ części i wpisuje ilość sztuk (liczba naturalna większa od zera), jaka ma zostać dodana do zamówienia.
      \item Pracownik zatwierdza operację.
      \item Powrót do widoku generowania zamówienia.
    \end{enumerate}
    \item Po wprowadzeniu wszystkich informacji o zamówieniu, pracownik zatwierdza operację.
    \item System prosi pracownika o potwierdzenie zamiaru wygenerowania zamówienia.
    \item W przypadku potwierdzenia zamiaru przez pracownika, system generuje plik PDF z zamówieniem na dostawę.
    \item Pracownik pobiera plik i wysyła go do dostawcy.
  \end{enumerate}


\end{enumerate}
\subsubsection{Opis przypadków użycia - pracownik}

Opis przypadków użycia dotyczących funkcjonalności związanych z zarządzaniem
pracownikami:

\begin{figure}[h!]
    \includegraphics[width=\textwidth,
    height=0.5\textheight]{graphics/UseCase/Pracownik/UseCaseDiagram.png}
  \caption{Diagram przypadków użycia związanych z procesowaniem danych
  pracowników}
\end{figure}

\begin{enumerate}
	
	  \item Dodawanie pracownika \\
  
  Opis słowny - ten przypadek użycia wspiera proces rozwoju firmy poprzez
  zatrudnianie nowych pracowników. Dane na temat wszystkich osób związanych ze
  sklepem są lepiej wykorzystywane jeśli są zarządzane przez system
  informatyczny. Aby uruchomić tą procedurę należy mieć specjalne uprawnienia
  jakie posiadają wyznaczeni pracownicy, czyli Kierownicy.
  
  \begin{longtable}{|p{5cm}|p{7cm}|}
 	\hline
	\textbf{Aktor} & Kierownik \\
	\hline
	\textbf{Warunki początkowe} & Kierownik zalogowany, posiadający wszelkie dane
	nowego pracownika\\
	\hline
	\textbf{Opis przebiegu interakcji} & Wybór panelu zarządzania sklepem,
	wypełnienie danych pracownika i potwierdzenie zapisu \\
	\hline
	\textbf{Sytuacje wyjątkowe} & Błędne dane, dany pracownik już zarejestrowany \\
	\hline
	\textbf{Warunki końcowe} & Zarejestrowanie nowego pracownika w systemie \\
	\hline
 \end{longtable}
  
  \begin{tabularx}{\linewidth}{ c X }
  Aktor: & Kierownik \\
  \end{tabularx}
   \begin{enumerate}
    \item Kierownik uruchamia stronę internetową panelu zarządzania sklepem
    i~wybiera opcję rejestracji.
    \item Kierownik wprowadza dane osobowe zatrudnianego pracownika.
    \item System sprawdza wstawione dane (czy istnieje już zarejestrowany
    w~systemie użytkownik, czy istnieje podany adres e-mail itp.)
    \item System wysyła na adres e-mail pracownika podany przez~kierownika
    wiadomość powitalną wraz z~linkiem umożliwiającym aktywowanie konta oraz
    ustalenie hasła.
    \item W ciągu określonego, zdefiniowanego czasu pracownik odwiedza stronę
    o~adresie przesłanym w~wiadomości powitalnej i ustala hasło dla konta.
  \end{enumerate}

	Dodawanie pracownika - scenariusz alternatywny - błąd danych
	\begin{enumerate}
	  \item Kroki 1-3 scenariusza głównego
	  \item System wyświetla komunikat informujący o miejscu oraz typie błędu w
	  wprowadzonych danych
	  \item Dane zostają poprawione
	  \item Kroki następne scenariusza głównego od 3 włącznie.
	\end{enumerate}
	
	Dodawanie pracownika - scenariusz alternatywny - pracownik już zarejestrowany
	\begin{enumerate}
	  \item Kroki 1-3 scenariusza głównego
	  \item System oświadcza, że wprowadzone dane odpowiadają istniejącemu już
	  pracownikowi - wskazując pokrywające się informacje
	  \item Po akceptacji, formularz jest odrzucany.
	\end{enumerate}

	  \item Zwolnienie pracownika \\
  
  Opis słowny - współpraca z pracownikiem kiedyś dobiega końca, w takiej
  sytuacji uprawniona osoba musi usunąć takiego pracownika z systemu poprzez
  opcję zwolnienia. Informacje o tej osobie nie są usuwane bezpowrotnie lecz
  archiwizowane zgodnie z przepisami aktualnego prawa.
  
  \begin{longtable}{|p{5cm}|p{7cm}|}
 	\hline
	\textbf{Aktor} & Kierownik \\
	\hline
	\textbf{Warunki początkowe} & Kierwonik zalogowany	\\
	\hline
	\textbf{Opis przebiegu interakcji} & Wyświetlenie listy pracowników,
	zaznaczenia konkretnej osoby, wybór opcji zwolnienia \\
	\hline
	\textbf{Sytuacje wyjątkowe} & Brak \\
	\hline
	\textbf{Warunki końcowe} & Wybrany pracownik nie pojawia się więcej na liście
	pracowników, jego dane są zarchiwizowane. 	\\
	\hline
 \end{longtable}
  
  \begin{tabularx}{\linewidth}{ c X }
  Aktor: & Kierownik \\
  \end{tabularx}
   \begin{enumerate}
    \item Kierownik uruchamia stronę internetową panelu zarządzania sklepem i~wybiera panel zarządzania pracownikami.
    \item Kierownik wyszukuje odpowiedniego pracownika.
    \item System wyświetla pracowników spełniających zadane kryteria wyszukiwania.
    \item Kierownik wybiera odpowiedniego pracownika.
    \item Kierownik wybiera opcję ,,Zwolnij''.
    \item System wyświetla formularz zwolnienia.
    \item Kierownik wypełnia formualrz podając przyczynę zwolnienia oraz datę od której pracownik ma być zwolniony.
    \item System sprawdza poprawność formualrza (np. czy można zwolnić pracownika w terminie wskazanym przez kierownika).
    \item W przypadku błędów system wyświetla odpowiedni komunikat, a kierownik poprawia dane w formularzu.
    \item System wyświetla prośbę o potwierdzenie operacji (dane pracownika oraz pytanie czy na pewno intencją kierownika
    było jego zwolnienie).
    \item Pracownik zatwierdza operację.
    \item System zapisuje informację o zwolnieniu pracownika.
    \item W momencie zaczęcia obowiązywania zwolnienia, system archiwizuje dane pracownika i~usuwa
    go~z~grupy zatrudnionych osób.
  \end{enumerate}

  \begin{figure}[H]
    \includegraphics[width=\textwidth,
    height=0.5\textheight]{graphics/UseCase/Pracownik/ZwolnieniePracownika.png}
  \caption{Diagram sekwencji dla przypadku użycia Zwolnienie pracownika -
  scenariusz główny}
\end{figure}  
	\item Ustalanie urlopu \\

	Opis słowny - w nowoczesnych firmach przyjaznych pracownikowi kwestia urlopu
	jest niezwykle ważna. Po pierwsze konieczne jest zapewnienie firmie rąk do
	pracy. Po drugie zadowolony pracownik jest najcennieszy, dlatego istotny jest
	kompromis pomiędzy dniami urlopowymi, a pracującymi (globalnie, dla
	wszystkich). Ta funkcja systemu umożliwia zgłaszanie chęci wykorzystania czasu
	urlopu - wyrażenie swoich preferencji urlopowych przez pracownika

  \begin{longtable}{|p{5cm}|p{7cm}|}
 	\hline
	\textbf{Aktor} & Pracownik \\
	\hline
	\textbf{Warunki początkowe} & Pracownik zalogowany, posiada niewykorzystane
	dni urlopowe\\
	\hline
	\textbf{Opis przebiegu interakcji} & Wybór panelu urlopów,
	zaznaczenie preferowanych terminów, oczekiwanie na akceptację \\
	\hline
	\textbf{Sytuacje wyjątkowe} & Niezgodność liczby dni urlopu w stosunku do
	zaznaczonego okresu preferowanego urlopu - wymagana korekta
	\\
	\hline
	\textbf{Warunki końcowe} & Wysłanie preferencji urlopowych do akceptacji przez
	kierownika
	\\
	\hline
 \end{longtable}
  

  \begin{tabularx}{\linewidth}{ c X }
  Aktor: & Pracownik \\
  \end{tabularx}
  \begin{enumerate}
    \item Pracownik uruchamia aplikację internetową sklepu i loguje się do systemu.
    \item Pracownik wybiera panel urlopów.
    \item System informuje pracownika o ilości dni urlopowych pozostałych do wykorzystania.
    \item Pracownik dodaje do kalendarza firmowego nowe żądanie urlopu.
    \item Pracownik uzupełnia dane dotyczące czasu przebywania na urlopie (datę początkową oraz datę końcową).
    \item System sprawdza, czy żądanie pracownika jest poprawne (np. czy pracownik może wziąć tak długi urlop).
    Jeśli nie, pracownik jest informowany o przyczynie błędu i musi ponownie wypełnić dane o urlopie.
    \item System zapisuje żądanie urlopu i informuje użytkownika o zmianie statusu żądania
    na~,,Oczekiwanie na odpowiedź kierownika''.
    \item Kierownik przegląda żądanie zgodnie ze scenariuszem ,,Przeglądanie żądań urlopowych''.
    \item Pracownik jest informowany o rozpatrzeniu żądania.
  \end{enumerate} 
  

	  \item Przeglądanie żądań urlopowych \\
  
  Opis słowny - preferencje odnośnie terminu wykorzystania zasłużonego urlopu
  przez pracowników spływają do Kierownika i wymagają jego akceptacji. Ten
  przypadek użycia nakreśla procedurę jaką należy wykonać w tym celu.
  
  \begin{tabularx}{\linewidth}{ c X }
  Aktor: & Kierownik \\
  Opis: & Możliwość przeglądania i rozpatrywania żądań urlopowych napływających od pracowników.\\
  \end{tabularx}
  \begin{enumerate}
    \item Kierownik loguje się do aplikacji internetowej systemu.
    \item Kierownik wybiera panel zarządzania pracownikami i przechodzi do sekcji ,,Urlopy''.
    \item System wyświetla kalendarz, w którym umieszczone są terminy urlopów.
    \item Kierownik ustawia odpowiednie filtry urlopów (np. wyświetlanie tylko nierozpatrzonych żądań).
    \item Kierownik wybiera jeden z urlopów i zmienia jego status.
    \item System sprawdza czy zmiana statusu jest poprawna (np. czy nie wystąpiła zmiana statusu z ,,Zrealizowany'' na ,,Odwołany'').
    \item W przypadku błędnej operacji system informuje o tym kierownika.
  \end{enumerate}
	
\end{enumerate}
\newpage
\subsection{Wymagania niefunkcjonalne}

\begin{enumerate}
  \item System powinien mieć możliwość przechowywania danych o 100 tys.
  użytkowników 
  \item System powinien obsługiwać bez znaczącego spadku wydajności 400
  użytkowników jednocześnie 
  \item System powinien być dostępny dla klientów 24 godziny na dobę 7 dni w
  tygodniu (możliwe są przerwy konserwacyjne, jednak nie dłuższe niż 4 godziny na miesiąc pracy) 
  \item System powinien umożliwiać klientom dostęp z dowolnego miejsca na
  świecie za pomocą sieci Internet 
  \item Klient powinien mieć dostęp do wszystkich swoich danych 
  \item Dane te powinny być chronione w zależności od ich tajności (hasło -
  dostępne tylko w postaci wartości funkcji skrótu; adres, e-mail - dostępne konkretnemu klientowi i pracownikom) 
  \item Komunikacja pomiędzy klientem (przeglądarką internetową, aplikacją
  mobilną itp.) powinna być szyfrowana w sposób uniemożliwiający odczytanie czułych informacji
  \item Autoryzacja użytkowników powinna odbywać się za pomocą loginu i hasła
  znanych tylko konkretnemu użytkownikowi 
  \item System powinien nadawać użytkownikowi uprawnienia niezbędne mu do
  poprawnego zamawiania produktów i zarządzania swoim kontem, jednak nie większe 
  \item System powinien umożliwiać automatyczne wysyłanie klientowi wiadomości
  e-mail (z prośbą o potwierdzenie zmiany hasła czy akceptacji warunków rejestracji)
  \item System powinien umożliwiać użytkownikowi zmianę (w ograniczonym stopniu)
  już złożonego zamówienia (zmiana adresu przed wysyłką itp.) bez konieczności ingerencji pracownika sklepu 
  \item System powinien przechowywać dane o klientach przez co najmniej 30 dni
  po wyrejestrowaniu lub usunięciu klienta (czas ten może się zmienić z powodów prawnych)
  
\end{enumerate}
