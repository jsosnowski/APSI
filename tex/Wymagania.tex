\newpage
\section{Wymagania}

% Dla wymagań funkcjonalnych i niefunkcjonalnych
% Globalnie zakładam, że zagnieżdżone listy będą numerowane cyframi
\setenumerate{label*=\arabic*.}

W tej sekcji znajduje się lista wymagań jakie spełniać powinien budowany system.
Podane są one z podziałem na dwie kategorie. Pierwsza to wymagania funkcjonalne
określające funkcjonalności systemu oraz sposoby ich użycia. Druga natomist to
wymagania niefunkcjonalne, które opisują ilościowe i jakościowe warunki
działania systemu.

\subsection{Wymagania funkcjonalne}
\subsubsection{Klient}

Wymagania funkcjonalne dotyczące zamówień realizowanych przez sklep:

\setenumerate{label*=\arabic*.}

\begin{enumerate}
  \item Dodanie nowego klienta
  \item Edycja danych klienta
  \begin{enumerate}
    \item Edycja adresu klienta
    \item Edycja adresu e-mail
  \end{enumerate}
  \item Edycja czułych danych klienta
  \begin{enumerate}
    \item Edycja hasła
    \item Edycja statusu (stały klient, nowy klient)
  \end{enumerate}
  \item Wyrejestrowanie się klienta
  \item Usunięcie klienta
\end{enumerate}

Opis przypadków użycia:
\subsubsection{Zamówienia}

Wymagania funkcjonalne dotyczące zamówień realizowanych przez sklep:

\setenumerate{label*=\arabic*.}

\begin{enumerate}
  \item Prezentacja zamówień
  \item Edycja, modyfikacja
  \begin{enumerate}
    \item Dodanie lub usunięcie produktu z zamówienia
    \item Zmiana ilości produktu
    \item Zmiana danych zamawiającego
  \end{enumerate}
  \item Usunięcie zamówienia w całości
  \item Wybór formy płatności
  \begin{enumerate}
    \item Płatność gotówką
    \begin{enumerate}
      \item Koszt w złotówkach
      \item Koszt w euro
      \item Koszt w wirtualnej walucie
    \end{enumerate}
    \item Płatność przelewem
    \item Płatność ratalna oparta o system szybkich pożyczek SuperBank
    \item Możliwość wpłaty zaliczki przed wysyłką
    \item Obniżenie kosztu o naliczone rabaty i zniżki
  \end{enumerate}
  \item Wybór sposobu potwierdzenia zamówienia (faktura, paragon)
  \item Generowanie faktury pro-forma dla danego zmówienia
  \item Zarządzanie terminem dostawy
  \item Ustawianie aktualnego stanu zamówienia.
\end{enumerate}

\pagebreak
\subsubsection{Opis przypadków użycia - klient}

Opis przypadków użycia wyjaśniające funkcjonalności związane z zarządzaniem
klientami:

\begin{enumerate}
  \item Rejestracja klienta \\
  \begin{tabularx}{\linewidth}{ c X }
  Aktor: & Klient \\
  Opis: & Możliwość rejestracji nowego klienta.\\
  \end{tabularx}
   \begin{enumerate}
    \item Klient uruchamia stronę internetową sklepu i wybiera opcję rejestracji
    \item Klient wstawia swoje dane osobowe i wybiera domyślny model płatności
    (kartą, za pobraniem itp.)
    \item System sprawdza wstawione dane (takie same hasła, czy istnieje już
    zarejestrowany w systemie użytkownik, czy istnieje podany adres e-mail itp.)
    \item System wysyła e-mail powitalny na adres podany przez klienta
    \item W ciągu określonego, zdefiniowanego czasu klient wybiera przesłany w
    e-mailu link, stają się pełnoprawnym użytkownikiem sklepu
  \end{enumerate}
  \item Złożenie zamówienia \\
  \begin{tabularx}{\linewidth}{ c X }
  Aktor: & Klient \\
  Opis: & Przedstawienie sposobu złożenia zamówienia.\\
  \end{tabularx}
  \begin{enumerate}
    \item Klient uruchamia stronę internetową sklepu i wyszukuje interesujące go
    produkty
    \item W momencie znalezienia pasującego produktu użytkownik wybiera opcję
    dodania do koszyka
    \item Po zakończeniu wyszukiwania użytkownik wybiera opcję przejścia do kasy
    \item System sprawdza, czy użytkownik jest zalogowany. Jeśli nie, procesuje
    przypadek użycia Logowanie do Systemu
    \item System sprawdza, czy użytkownik jest stałym klientem. Jeśli tak,
    dolicza rabat do ustalonej ceny (do sumy cen poszczególnych produktów)
    \item Użytkownik wybiera sposób płatności
    \item System dodaje do wcześniej ustalonej ceny koszty wynikające ze sposobu
    płatności
    \item Użytkownik wybiera sposób dostawy (poczta, kurier, odbiór osobisty
    itp.)
    \item System dodaje do ceny koszty wynikające ze sposobu dostawy
    \item Użytkownik, po sprawdzeniu wszystkich danych, decyduje się na złożenie
    zamówienia - po tym momencie nie może już ono być cofnięte
    \item System wysyła do użytkownika e-mail potwierdzający wraz z przewidywaną
    datą realizacji zamówienia
  \end{enumerate} 
  \item Edycja danych klienta \\
  \begin{tabularx}{\linewidth}{ c X }
  Aktor: & Klient \\
  Opis: & Możliwość zmiany, uzupełnienia danych osobowych klienta.\\
  \end{tabularx}
  \begin{enumerate}
    \item Klient uruchamia witrynę internetową sklepu
    \item Klient loguje się do systemu (tylko osoba zalogowana może zmieniać
    swoje dane)
    \item Klient edytuje wybrane pozycje ze swojego opisu (adres, numer
    telefonu itp.)
    \item W przypadku zmiany hasła klient proszony jest o podanie starego jak i
    nowego (dwukrotnie) hasła
    \item Klient zatwierdza wprowadzone zmiany
    \item System wysyła na podany przez użytkownika adres e-mail (nowy, jeśli
    to adres e-mail był jedną ze zmienianych wartości) informację o zmianie.
  \end{enumerate}
  \item Wyrejestrowanie się klienta \\
  \begin{tabularx}{\linewidth}{ c X }
  Aktor: & Klient \\
  Opis: & Klient ma możliwość w każdym momencie usunąć swoje konto z systemu.\\
  \end{tabularx}
  \begin{enumerate}
    \item Klient uruchamia witrynę internetową i loguje się na swoje konto
    (przypadek użycia Logowanie Do Systemu)
    \item Klient wybiera opcję usunięcia danych
    \item System sprawdza, czy istnieją niezrealizowane (oczekujące) zamówienia.
    Jeśli tak, wyświetla się alert z informacją, czy dane zamówienie zostało już
    wcześniej opłacone
    \item Jeśli istniały już zamówienia, które zostały opłacone a nie zostały
    jeszcze zrealizowane, system zleca odesłanie określonej kwoty pieniężnej z
    powrotem na konto użytkownika (z pominięciem kosztów obsługi)
    \item Klient zostaje poproszony o podanie przyczyn swojej decyzji -
    wypełnianie jest nieobowiązkowe
    \item Dane przechowywane są przez następne 7 dni (wymaganie prawne). W tym
    czasie klient może ponownie zarejestrować się w systemie bez utraty
    poprzednich danych
    \item W przypadku braku ponownej rejestracji dane zostają na stałe usunięte
    z firmowej bazy danych
  \end{enumerate}
  \item Usunięcie klienta \\
  \begin{tabularx}{\linewidth}{ c X }
  Aktor: & Pracownik \\
  Opis: & Klienta można usunąć administracyjnie na przykład z powodów
  naruszenia regulaminu.\\
  \end{tabularx}
  \begin{enumerate}
    \item Pracownik sklepu wyszukuje klienta o konkretnym imieniu i nazwisku
    (lub według innych kryteriów)
    \item Pracownik wybiera opcję usunięcia klienta. 
    \item Pracownik wpisuje powód, dla którego usuwa użytkownika (informacja ta
    będzie przesłana do klienta w wiadomości e-mail)
    \item Pracownik wypełnia dane dotyczące kwestii niezrealizowanych zamówień i
    nieotrzymanych płatności
    \item Obie informacji (z poprzednich 2 kroków) są przekazywane na podany
    przez użytkownika adres e-mail
    \item Dane są przechowywane przez następne 30 dni - w tym czasie użytkownik
    może złożyć reklamację i ewentualnie odzyskać dostęp do konta
    \item Po 30 dniach, jeśli nie zostanie pozytywnie rozpatrzona prośba o
    przywrócenie konta, dane są na stałe usuwane z bazy danych
  \end{enumerate}
\end{enumerate}
\subsubsection{Opis przypadków użycia - zamówienia}

Przypadki użycia wyjaśniające funkcjonalności systemu związane z zarządzaniem
zamówieniami.

\begin{enumerate}
  \item Prezentacja zamówień\\
  \begin{tabularx}{\linewidth}{ c X }
  Aktor: & Pracownik \\
  Opis: & Prezentacja panelu z listą wszystkich zamówień znajdujących się w~systemie
  oraz możliwościami kontroli i zarządzania nimi.\\
  \end{tabularx}
	\begin{enumerate}
	  \item Pracownik loguje się w Panelu Zarządzania
	  \item Wybiera Panel Zarządzania Zamówieniami
	  \item Wyświetlana jest lista zamówień z możliwością modyfikacji widoków
	  oraz panelem opcji (wszystkie opisane w poniższych przypadkach użycia)
	\end{enumerate}

  \item Edycja, modyfikacja zamówień\\
  \begin{tabularx}{\linewidth}{c X}
  Aktor: & Pracownik \\
  Opis: & Funkcjonalność umożliwia modyfikację produktów w zamówieniu oraz
  zmianę danych odbiorcy.
  \end{tabularx}
	\begin{enumerate}
	  \item Pracownik po autoryzacji w panelu sterowania systemu, przechodzi do
	  panelu Zarządzania Zamówieniami (patrz Zamowienia przypadek użycia 1)
	  \item Z wyświetlonej przez system listy zamówień, pracownik wybiera jeden
	  element
	  \item W celu edycji produktów:
		\begin{enumerate}
		  \item Wybiera opcję podglądu zawartości zamówienia
		  \item Z wyświetlonej listy zamówionych produktów zaznacza jedną pozycję
		  \item Wybiera opcję edycji
		  \item Otrzymuje informacje o konkretnym produkcie (jego ID, szczegółowy opis)
		  oraz zamówioną ilość oraz podsumowanie (cenę, informację o udzielonych rabatach na dany produkt)
		  \item Pole z ilością produktu umożliwia modyfikację – wystarczy wprowadzić
		  liczbę z zakresu od 1 do maksymalnej liczby aktualnie dostępnych produktów w
		  magazynie (0 nie wchodzi w zakres bo do tego służy funkcja usunięcia)
		\end{enumerate}
	  \item W celu usunięcia produktu:
		\begin{enumerate}
		  \item Wybiera opcję podglądu zawartości zamówienia
		  \item Z wyświetlonej listy zamówionych produktów zaznacza jedną pozycję
		  \item Wybiera opcję Usuń
		  \item System pyta o potwierdzenie i po akceptacji dokonuje wykluczenia
		  produktu z zamówienia oraz wysyła powiadomienie do Zamawiającego
		\end{enumerate}
	  \item W celu dodania produktu:
		\begin{enumerate}
		  \item Wybiera opcję podglądu zawartości zamówienia
		  \item Wybiera opcję Dodaj produkt
		  \item Otworzony zostaje system zakupowy\\ 
		  (przebieg wyboru produktu - opisany w przypadkach użycia odnoszących się do
		  Produktów)
		  \item Po wybraniu produktu system wyświetla informację o tym jakie zostaną
		  wprowadzone zmiany i czeka na akceptację
		  \item Po akceptacji, zamówienie zostaje zmodyfikowane (produkt dodany),
		  koszt zaktualizowany oraz system informuje odbiorcę zamówienia (klienta) o
		  zaszłych zmianach – za pomocą wiadomości email (z ewentualną poprawioną
		  fakturą pro-forma, jeśli była zaznaczona taka opcja) 
	  \end{enumerate} %koniec alternatywy Dodania produktu
	\end{enumerate} %koniec UC Edycja, modyfikacja zamówień
  
  \item Zmiana danych zamawiającego\\
  \begin{tabularx}{\linewidth}{c X}
  Aktor: & Pracownik \\
  Opis: & Można zmienić dane odbiorcy na potrzeby danego zamówienia (zmiana
  danych tylko w ramach konkretnej faktury). Dotyczy to w szczególności adresu i
  danych osobowych osoby odpowiedzialnej za zamówienie.
  \end{tabularx}  
	\begin{enumerate}
	  \item Pracownik po autoryzacji w panelu sterowania systemu, przechodzi do
	  panelu Zarządzania Zamówieniami (patrz Zamówienia przypadek użycia 1)
	  \item Z wyświetlonej przez system listy zamówień, pracownik wybiera jeden
	  element i wybiera opcję Zmień Dane Odbiorcy
	  \item System prezentuje aktualnie dane odbiorcy (mogą to być aktualne dane
	  klienta, albo już wcześniej modyfikowane dane osobowe wprowadzone specjalnie w
	  ramach tego zamówienia)
	  \item Pracownik modyfikuje wybraną przez siebie składową danych (wszystkie
	  elementy pozwalają na edycję) i zatwierdza wprowadzone zmiany
	  \item System wyświetla zapytanie o potwierdzenie zmian i po jego akceptacji
	  wysyła powiadomienie do klienta o zaszłych zmianach – wiadomość drogą
	  elektroniczną
	\end{enumerate}

  \item Usunięcie zamówienia w całości\\
  \begin{tabularx}{\linewidth}{c X}
  Aktor: & Pracownik \\
  Opis: & Istnieje możliwość anulowania zamówienia – na życzenie klienta lub z
  powodów biznesowych sklepu.
  \end{tabularx}  
	\begin{enumerate}
	  \item Pracownik po autoryzacji w panelu sterowania systemu, przechodzi do
	  panelu Zarządzania Zamówieniami (patrz Zamówienia przypadek użycia 1)
	  \item Z wyświetlonej przez system listy zamówień, pracownik wybiera jeden
	  element i wybiera opcję Usuń zamówienie
	  \item System wyświetla ostrzeżenie (wraz ze szczegółową informacją o
	  zamówieniu) i pyta o potwierdzenie
	  \item Pracownik potwierdza chęć usunięcia danego zamówienia. Ma też możliwość
	  wpisania krótkiego uzasadnienia tej operacji
	  \item System dokonuje usunięcia oraz wysyła powiadomienie o anulowaniu
	  zamówienia do zamawiającego (drogą elektroniczną)
	\end{enumerate}

  \item Edycja formy płatności\\
  \begin{tabularx}{\linewidth}{c X}
  Aktor: & Pracownik \\
  Opis: & Pracownik ma możliwość zmiany początkowo wybranej formy płatności
  danego zamówienia. Odbywa się to na wniosek zamawiającego lub osoby
  odpowiedzialnej za zamówienia po stronie Sklepu.
  \end{tabularx}    
	\begin{enumerate}
	  \item Z listy zamówień pracownik wybiera jedno i wybiera opcję Zmiana formy
	  płatności
	  \item System prezentuje widok wyboru pomiędzy dostępnymi formami płatności
	  (specyfikacja w wymaganiach niefunkcjonalnych punkt \ref{itm:Platnosci})
	  \item Pracownik dokonuje wyboru formy oraz waluty.
	  \item System powiadamia klienta o zmianie formy płatności drogą elektroniczną.
	\end{enumerate}

  \item Wybór sposobu potwierdzenia zamówienia\\
  \begin{tabularx}{\linewidth}{c X}
  Aktor: & Pracownik \\
  Opis: & Możliwość zmiany sposobu udokumentowania przeprowadzonej transakcji
  (zazwyczaj będzie to faktura albo paragon). Powodem takich zmian mogą być
  nawet regulacje prawne.
  \end{tabularx}	
	\begin{enumerate}
	  \item Z listy zamówień pracownik wybiera jedno i wybiera opcję Zmiana
	  Potwierdzenia Transakcji
	  \item System prezentuje widok wyboru pomiędzy dostępnymi sposobami
	  potwierdzenia (udokumentowania) prowadzonej transakcji (wymagania
	  niefunkcyjne punkt \ref{itm:PotwierdzenieTransakcji})
	  \item Pracownik dokonuje wyboru oraz może uruchomić proces generacji
	  dokumentu.
	  \item W przypadku generacji dokumenty system wyświetla go pracownikowi.
	  \item Po akceptacji informacje o zmianie wraz z dokumentami wysyłane są drogą
	  elektroniczną do klienta.
	\end{enumerate}

  \item Generowanie faktury pro-forma dla danego zmówienia\\
  \begin{tabularx}{\linewidth}{c X}
  Aktor: & Pracownik \\
  Opis: & Możliwość utworzenie faktury pro-forma na podstawie danego zamówienia
  oraz przesłanie jej klientowi drogą elektroniczną lub wydruk.
  \end{tabularx}
	\begin{enumerate}
	  \item Z listy zamówień pracownik wybiera jedno i wybiera opcję Generuj
	  Pro-Forma
	  \item Dla wybranego zamówienia system generuje pełną fakturę po czym
	  prezentuje ją pracownikowi
	  \item Pracownik ma możliwość odrzucenia lub akceptacji dokumentu.
	  \item W przypadku akceptacji system wyświetla widok wyboru z opcjami wysyłki
	  do klienta.
	  \item Po wybraniu pożądanej opcji przez pracownika, system wysyła dokument do
	  klienta albo do drukarki.
	\end{enumerate}

  \item Zarządzanie dostawą\\
  \begin{tabularx}{\linewidth}{c X}
  Aktor: & Pracownik \\
  Opis: & Termin realizacji zamówienia oraz sposób dostawy mogą być
  modyfikowany dowolnie w zależności od możliwości biznesowych Sklepu i
  aktualnego stanu zamówienia.
  \end{tabularx}
	\begin{enumerate}
	  \item Z listy zamówień użytkownik wybiera jedno i wybiera opcję Edycji
	  Dostawy
	  \item System prezentuje informacje o wybranym sposobie i terminie dostawy
	  \item Użytkownik wybiera opcję Zmiany daty realizacji
	  \item System prezentuje widok kalendarza z zaznaczoną dotychczasową datą
	  realizacji.
	  \item Użytkownik przesuwa datę realizacji projektu i ma możliwość podania
	  wiadomości wyjaśniającej modyfikację.
	  \item Użytkownik wybiera opcję Zmiany sposobu dostawy
	  \item System wyświetla wszystkie aktualnie dostępne opcje razem ze
	  szczegółami (cena, średni czas)
	  \item Użytkownik dokonuje wyboru środka transportu i zatwierdza zmiany
	  \item System aktualizuje koszt całego zamówienia uwzględniając kwotę
	  transportu oraz wysyła powiadomienie o zmianach (termin lub/i sposób dostawy)
	  do klienta wraz z informacją wyjaśniającą wpisaną przez pracownika.
	\end{enumerate}

  \item Ustawianie aktualnego stanu zamówienia\\
  \begin{tabularx}{\linewidth}{c X}
  Aktor: & Pracownik \\
  Opis: & Zamówienie może znajdować się w pewnych stanach realizacji (np. w
  przygotowaniu, w realizacji, wysłane - konkretne stany określają wymagania
  niefunkcjonalne). Istnieje możliwość zmiany aktualnego stanu zamówienia.
  \end{tabularx}
	\begin{enumerate}
	  \item Z listy zamówień pracownik wybiera jedno i wybiera opcję Zmień Stan
	  \item System prezentuje widok z dostępnymi stanami dla danego zamówienia
	  \item Pracownik dokonuje wyboru i zatwierdza zmiany.
	  \item Jeśli pracownik wybiera opcję Powiadom, to system powiadamia klienta o
	  zmianie stanu jaka nastąpiła i przesyła krótkie wyjaśnienie.
	\end{enumerate}
	 
\end{enumerate}

\newpage
\subsection{Wymagania niefunkcjonalne}

\begin{enumerate}
  \item System powinien mieć możliwość przechowywania danych o 100 tys.
  użytkowników 
  \item System powinien obsługiwać bez znaczącego spadku wydajności 400
  użytkowników jednocześnie 
  \item System powinien być dostępny dla klientów 24 godziny na dobę 7 dni w
  tygodniu (możliwe są przerwy konserwacyjne, jednak nie dłuższe niż 4 godziny na miesiąc pracy) 
  \item System powinien umożliwiać klientom dostęp z dowolnego miejsca na
  świecie za pomocą sieci Internet 
  \item Klient powinien mieć dostęp do wszystkich swoich danych 
  \item Dane te powinny być chronione w zależności od ich tajności (hasło -
  dostępne tylko w postaci wartości funkcji skrótu; adres, e-mail - dostępne konkretnemu klientowi i pracownikom) 
  \item Komunikacja pomiędzy klientem (przeglądarką internetową, aplikacją
  mobilną itp.) powinna być szyfrowana w sposób uniemożliwiający odczytanie czułych informacji
  \item Autoryzacja użytkowników powinna odbywać się za pomocą loginu i hasła
  znanych tylko konkretnemu użytkownikowi 
  \item System powinien nadawać użytkownikowi uprawnienia niezbędne mu do
  poprawnego zamawiania produktów i zarządzania swoim kontem, jednak nie większe 
  \item System powinien umożliwiać automatyczne wysyłanie klientowi wiadomości
  e-mail (z prośbą o potwierdzenie zmiany hasła czy akceptacji warunków rejestracji)
  \item System powinien umożliwiać użytkownikowi zmianę (w ograniczonym stopniu)
  już złożonego zamówienia (zmiana adresu przed wysyłką itp.) bez konieczności ingerencji pracownika sklepu 
  \item System powinien przechowywać dane o klientach przez co najmniej 30 dni
  po wyrejestrowaniu lub usunięciu klienta (czas ten może się zmienić z powodów prawnych)
  
\end{enumerate}
