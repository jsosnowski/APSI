\newpage
\section{Wymagania}

% Dla wymagań funkcjonalnych i niefunkcjonalnych
% Globalnie zakładam, że zagnieżdżone listy będą numerowane cyframi
\setenumerate{label*=\arabic*.}

W tej sekcji znajduje się lista wymagań jakie spełniać powinien budowany system.
Podane są one z podziałem na dwie kategorie. Pierwsza to wymagania funkcjonalne
określające funkcjonalności systemu oraz sposoby ich użycia. Druga natomist to
wymagania niefunkcjonalne, które opisują ilościowe i jakościowe warunki
działania systemu.

\subsection{Wymagania funkcjonalne}
\subsubsection{Zamówienia}

Wymagania funkcjonalne dotyczące zamówień realizowanych przez sklep:

\begin{enumerate}
  \item Prezentacja zamówień
  \item Edycja, modyfikacja
  \begin{enumerate}
    \item Dodanie lub usunięcie produktu z zamówienia
    \item Zmiana ilości produktu
  \end{enumerate}
  \item Zmiana danych zamawiającego
  \item Usunięcie zamówienia w całości
  \item Edycja formy płatności
  \begin{enumerate}
    \item Płatność gotówką
    \begin{enumerate}
      \item Koszt w złotówkach
      \item Koszt w euro
      \item Koszt w wirtualnej walucie
    \end{enumerate}
    \item Płatność przelewem
    \item Płatność ratalna oparta o system szybkich pożyczek SuperBank
    \item Możliwość wpłaty zaliczki przed wysyłką
    \item Obniżenie kosztu o naliczone rabaty i zniżki
  \end{enumerate}
  \item Wybór sposobu potwierdzenia zamówienia (faktura, paragon)
  \item Generowanie faktury pro-forma dla danego zmówienia
  \item Zarządzanie terminem dostawy
  \item Ustawianie aktualnego stanu zamówienia.
\end{enumerate}
\subsubsection{Klient}

Wymagania funkcjonalne dotyczące klientów zamawiających części w sklepie

\begin{enumerate}
  \item Dodanie nowego klienta
  \item Edycja danych klienta
  \begin{enumerate}
    \item Edycja adresu klienta
    \item Edycja adresu e-mail
  \end{enumerate}
  \item Edycja czułych danych klienta
  \begin{enumerate}
    \item Edycja hasła
    \item Edycja statusu (stały klient, nowy klient)
  \end{enumerate}
  \item Wyrejestrowanie się klienta
  \item Usunięcie klienta
\end{enumerate}

 
\subsubsection{Opis przypadków użycia - klient}

Opis przypadków użycia wyjaśniające funkcjonalności związane z zarządzaniem
klientami:

\begin{enumerate}
  \item Rejestracja klienta \\
  \begin{tabularx}{\linewidth}{ c X }
  Aktor: & Klient \\
  Opis: & Możliwość rejestracji nowego klienta.\\
  \end{tabularx}
   \begin{enumerate}
    \item Klient uruchamia stronę internetową sklepu i wybiera opcję rejestracji
    \item Klient wstawia swoje dane osobowe i wybiera domyślny model płatności
    (kartą, za pobraniem itp.)
    \item System sprawdza wstawione dane (takie same hasła, czy istnieje już
    zarejestrowany w systemie użytkownik, czy istnieje podany adres e-mail itp.)
    \item System wysyła e-mail powitalny na adres podany przez klienta
    \item W ciągu określonego, zdefiniowanego czasu klient wybiera przesłany w
    e-mailu link, stając się pełnoprawnym użytkownikiem sklepu
  \end{enumerate}
  \item Złożenie zamówienia \\
  \begin{tabularx}{\linewidth}{ c X }
  Aktor: & Klient \\
  Opis: & Przedstawienie sposobu złożenia zamówienia.\\
  \end{tabularx}
  \begin{enumerate}
    \item Klient uruchamia stronę internetową sklepu i wyszukuje interesujące go
    produkty
    \item W momencie znalezienia pasującego produktu użytkownik wybiera opcję
    dodania do koszyka
    \item Po zakończeniu wyszukiwania użytkownik wybiera opcję przejścia do kasy
    \item System sprawdza, czy użytkownik jest zalogowany. Jeśli nie, procesuje
    przypadek użycia Logowanie do Systemu
    \item System sprawdza, czy użytkownik jest stałym klientem. Jeśli tak,
    dolicza rabat do ustalonej ceny (do sumy cen poszczególnych produktów)
    \item Użytkownik wybiera sposób płatności
    \item System dodaje do wcześniej ustalonej ceny koszty wynikające ze sposobu
    płatności
    \item Użytkownik wybiera sposób dostawy (poczta, kurier, odbiór osobisty
    itp.)
    \item System dodaje do ceny koszty wynikające ze sposobu dostawy
    \item Użytkownik, po sprawdzeniu wszystkich danych, decyduje się na złożenie
    zamówienia - po tym momencie nie może już ono być cofnięte
    \item System wysyła do użytkownika e-mail potwierdzający wraz z przewidywaną
    datą realizacji zamówienia
  \end{enumerate} 
  \item Edycja danych klienta \\
  \begin{tabularx}{\linewidth}{ c X }
  Aktor: & Klient \\
  Opis: & Możliwość zmiany, uzupełnienia danych osobowych klienta.\\
  \end{tabularx}
  \begin{enumerate}
    \item Klient uruchamia witrynę internetową sklepu
    \item Klient loguje się do systemu (tylko osoba zalogowana może zmieniać
    swoje dane)
    \item Klient edytuje wybrane pozycje ze swojego opisu (adres, numer
    telefonu itp.)
    \item W przypadku zmiany hasła klient proszony jest o podanie starego jak i
    nowego (dwukrotnie) hasła
    \item Klient zatwierdza wprowadzone zmiany
    \item System wysyła na podany przez użytkownika adres e-mail (nowy, jeśli
    to adres e-mail był jedną ze zmienianych wartości) informację o zmianie.
  \end{enumerate}
  \item Wyrejestrowanie się klienta \\
  \begin{tabularx}{\linewidth}{ c X }
  Aktor: & Klient \\
  Opis: & Klient ma możliwość w każdym momencie usunąć swoje konto z systemu.\\
  \end{tabularx}
  \begin{enumerate}
    \item Klient uruchamia witrynę internetową i loguje się na swoje konto
    (przypadek użycia Logowanie Do Systemu)
    \item Klient wybiera opcję usunięcia danych
    \item System sprawdza, czy istnieją niezrealizowane (oczekujące) zamówienia.
    Jeśli tak, wyświetla się alert z informacją, czy dane zamówienie zostało już
    wcześniej opłacone
    \item Jeśli istniały już zamówienia, które zostały opłacone a nie zostały
    jeszcze zrealizowane, system zleca odesłanie określonej kwoty pieniężnej z
    powrotem na konto użytkownika (z pominięciem kosztów obsługi)
    \item Klient zostaje poproszony o podanie przyczyn swojej decyzji -
    wypełnianie jest nieobowiązkowe
    \item Dane przechowywane są przez Okres Przechowywania Danych (wymaganie
    prawne - patrz Wymagania niefunkcjonalne punkt \ref{itm:OPD}). W tym
    czasie klient może ponownie zarejestrować się w systemie bez utraty poprzednich danych
    \item W przypadku braku ponownej rejestracji dane zostają na stałe usunięte
    z firmowej bazy danych
  \end{enumerate}
  \item Usunięcie klienta \\
  \begin{tabularx}{\linewidth}{ c X }
  Aktor: & Pracownik \\
  Opis: & Klienta można usunąć administracyjnie na przykład z powodów
  naruszenia regulaminu.\\
  \end{tabularx}
  \begin{enumerate}
    \item Pracownik sklepu wyszukuje klienta o konkretnym imieniu i nazwisku
    (lub według innych kryteriów)
    \item Pracownik wybiera opcję usunięcia klienta. 
    \item Pracownik wpisuje powód, dla którego usuwa użytkownika (informacja ta
    będzie przesłana do klienta w wiadomości e-mail)
    \item Pracownik wypełnia dane dotyczące kwestii niezrealizowanych zamówień i
    nieotrzymanych płatności
    \item Obie informacji (z poprzednich 2 kroków) są przekazywane na podany
    przez użytkownika adres e-mail
    \item Dane są przechowywane przez Okres Magazynowania Danych (patrz
    Wymagania Niefunkcjonalne punkt \ref{itm:OMD}) - w tym czasie użytkownik
    może złożyć reklamację i ewentualnie odzyskać dostęp do konta
    \item Po tym czasie, jeśli prośba o przywrócenie konta nie zostanie
    pozytywnie rozpatrzona, dane są na stałe usuwane z systemu
  \end{enumerate}
\end{enumerate}
\newpage
\subsection{Wymagania niefunkcjonalne}

\begin{enumerate}
  \item Pojemność: \\ 
  System powinien mieć możliwość przechowywania danych o 100 tys.
  użytkowników 
  \item Wydajność: \\ 
  System powinien obsługiwać bez znaczącego spadku wydajności 400
  użytkowników ``jednocześnie''. Zakładając, że użytkownik będzie wymagał
  maksymalnie 20 odświeżeń widoku systemu na minutę (jedna podstrona na 3
  sekundy). System powinien działać z wydajnością 8000 odświeżeń/minutę. 
  \item System powinien być dostępny dla klientów 24 godziny na dobę 7 dni w
  tygodniu (możliwe są przerwy konserwacyjne, jednak nie dłuższe niż 4 godziny na miesiąc pracy) 
  \item Średni czas naprawy (MTTR - ang. Mean Time to Recover) na poziomie
  1~godziny
  \item System powinien umożliwiać klientom dostęp z dowolnego miejsca na
  świecie za pomocą sieci Internet oraz jego działanie powinno być niezależne od
  używanej platformy systemowo-sprzętowej użytkownika.
  \item Dane osobowe muszą być przetwarzane zgodnie z ustawą o ochronie danych
  osobowych z dnia 29 sierpnia 1997 r. 
  \item Klient powinien mieć dostęp do wszystkich swoich danych (łącznie z
  możliwością ich aktualizacji i usunięcia) zgodnie z polskim prawem
  \item Dane te powinny być chronione w zależności od ich poziomu poufności
  (dane do autoryzacji powinny być zabezpieczone przed możliwością odczytu nawet
  przez administratora) 
  \item Komunikacja pomiędzy klientem (przeglądarką internetową, aplikacją
  mobilną itp.) powinna być szyfrowana w sposób uniemożliwiający odczytanie czułych informacji
  \item System powinien mieć wbudowane procedury przeciwdziałania sytuacjom
  awaryjnym - procedury uruchamiane przez administratora
  \begin{enumerate}
    \item Procedury sprawdzenia spójności danych - po odzyskaniu sprawności, np.
    po awarii sprzętu
    \item Procedury uruchamiane w przypadku wykrycia włamania (między innymi,
    odłączenie systemu od sieci Internet, zablokowanie modyfikacji elementów
    systemu itp.)
  \end{enumerate} 
  \item System posiadać będzie hierarchię uprawnień (ról) dla użytkowników,
  przydzielanych im w celu umożliwienia korzystania z dodatkowych
  funkcjonalności
  \item System domyślnie powinien nadawać użytkownikowi uprawnienia nie większe
  niż niezbędne mu do poprawnego zamawiania produktów i zarządzania swoim
  kontem
  \item System powinien umożliwiać automatyczne wysyłanie klientowi wiadomości
  e-mail (z prośbą o potwierdzenie zmiany hasła czy akceptacji warunków rejestracji)
  \item System powinien umożliwiać użytkownikowi zmianę (w ograniczonym stopniu)
  już złożonego zamówienia (zmiana adresu przed wysyłką itp.) bez konieczności
  ingerencji pracownika sklepu
  \item System powinien być zdolny do wyświetlania informacji w wielu językach.
  Początkowo będzie to język polski i angielski. Istnieje jednak możliwość
  rozszerzenia o kolejne.
  \item \label{itm:OPD} Okres Przechowywania Danych - to czas przez który będą
  przechowywane dane użytkownika sprzed ich zmiany lub wyrejestrowania - system zapewnia
  magazynowanie tych danych co najmniej przez 7 dni.
  \item \label{itm:OMD} Okres Magazynowania Danych to czas 30 dni przez które
  system powinien przechowywać dane o klientach po usunięciu konta klienta (czas ten może się
  zmienić z powodów prawnych)
  \item \label{itm:Platnosci} System wspiera następujące formy płatności:
  \begin{enumerate}
    \item Płatność gotówką
    \item Przelew bankowy
    \item Płatność ratalna w oparciu o zewnętrzną usługę bankową
  \end{enumerate}
  Waluty: polski złoty, euro, bitcoin
  \item \label{itm:PotwierdzenieTransakcji} Forma potwierdzenia transakcji:
  faktura albo paragon
  \\
  System powinien umożliwiać generowanie tych dokumentów oraz ich wydruk.
  
\end{enumerate}
